\documentclass[a4paper,12pt,twoside]{memoir}

% Castellano
\usepackage[spanish,es-tabla]{babel}
\selectlanguage{spanish}
\usepackage[utf8]{inputenc}
\usepackage[T1]{fontenc}
\usepackage{lmodern} % Scalable font
\usepackage{microtype}
\usepackage{placeins}
\usepackage{float}
\usepackage{adjustbox}
\usepackage{caption}
\usepackage{subcaption}
\usepackage{listings}
\usepackage[
backend=biber,
style=numeric,
sorting=none
]{biblatex}
\addbibresource{bibliografia.bib}

\RequirePackage{booktabs}
\RequirePackage[table]{xcolor}
\RequirePackage{xtab}
\RequirePackage{multirow}

% Links
\PassOptionsToPackage{hyphens}{url}\usepackage[colorlinks]{hyperref}
\hypersetup{
	allcolors = {red}
}

% Ecuaciones
\usepackage{amsmath}

\DeclareCaptionType{equ}[][]

% Rutas de fichero / paquete
\newcommand{\ruta}[1]{{\sffamily #1}}

% Párrafos
\nonzeroparskip

% Huérfanas y viudas
\widowpenalty100000
\clubpenalty100000

% Imágenes

% Comando para insertar una imagen en un lugar concreto.
% Los parámetros son:
% 1 --> Ruta absoluta/relativa de la figura
% 2 --> Texto a pie de figura
% 3 --> Tamaño en tanto por uno relativo al ancho de página
\usepackage{graphicx}
\newcommand{\imagen}[3]{
	\begin{figure}[!h]
		\centering
		\includegraphics[width=#3\textwidth]{#1}
		\caption{#2}\label{fig:#1}
	\end{figure}
	\FloatBarrier
}

% Comando para insertar una imagen sin posición.
% Los parámetros son:
% 1 --> Ruta absoluta/relativa de la figura
% 2 --> Texto a pie de figura
% 3 --> Tamaño en tanto por uno relativo al ancho de página
\newcommand{\imagenflotante}[3]{
	\begin{figure}
		\centering
		\includegraphics[width=#3\textwidth]{#1}
		\caption{#2}\label{fig:#1}
	\end{figure}
}

% El comando \figura nos permite insertar figuras comodamente, y utilizando
% siempre el mismo formato. Los parametros son:
% 1 --> Porcentaje del ancho de página que ocupará la figura (de 0 a 1)
% 2 --> Fichero de la imagen
% 3 --> Texto a pie de imagen
% 4 --> Etiqueta (label) para referencias
% 5 --> Opciones que queramos pasarle al \includegraphics
% 6 --> Opciones de posicionamiento a pasarle a \begin{figure}
\newcommand{\figuraConPosicion}[6]{%
  \setlength{\anchoFloat}{#1\textwidth}%
  \addtolength{\anchoFloat}{-4\fboxsep}%
  \setlength{\anchoFigura}{\anchoFloat}%
  \begin{figure}[#6]
    \begin{center}%
      \Ovalbox{%
        \begin{minipage}{\anchoFloat}%
          \begin{center}%
            \includegraphics[width=\anchoFigura,#5]{#2}%
            \caption{#3}%
            \label{#4}%
          \end{center}%
        \end{minipage}
      }%
    \end{center}%
  \end{figure}%
}

%
% Comando para incluir imágenes en formato apaisado (sin marco).
\newcommand{\figuraApaisadaSinMarco}[5]{%
  \begin{figure}%
    \begin{center}%
    \includegraphics[angle=90,height=#1\textheight,#5]{#2}%
    \caption{#3}%
    \label{#4}%
    \end{center}%
  \end{figure}%
}
% Para las tablas
\newcommand{\otoprule}{\midrule [\heavyrulewidth]}
%
% Nuevo comando para tablas pequeñas (menos de una página).
\newcommand{\tablaSmall}[5]{%
 \begin{table}
  \begin{center}
   \rowcolors {2}{gray!35}{}
   \begin{tabular}{#2}
    \toprule
    #4
    \otoprule
    #5
    \bottomrule
   \end{tabular}
   \caption{#1}
   \label{tabla:#3}
  \end{center}
 \end{table}
}

% Nuevo comando para tablas pequeñas (menos de una página).
\newcommand{\tablaSmallFija}[5]{%
 \begin{table}[H]
  \begin{center}
   \rowcolors {2}{gray!35}{}
   \begin{tabular}{#2}
    \toprule
    #4
    \otoprule
    #5
    \bottomrule
   \end{tabular}
   \caption{#1}
   \label{tabla:#3}
  \end{center}
 \end{table}
}

%
% Nuevo comando para tablas pequeñas (menos de una página).
\newcommand{\tablaSmallSinColores}[5]{%
 \begin{table}
  \begin{center}
   \begin{tabular}{#2}
    \toprule
    #4
    \otoprule
    #5
    \bottomrule
   \end{tabular}
   \caption{#1}
   \label{tabla:#3}
  \end{center}
 \end{table}
}

% Nuevo comando para tablas pequeñas (menos de una página).
\newcommand{\tablaSmallSinColoresFija}[5]{%
 \begin{table}[H]
  \begin{center}
   \begin{tabular}{#2}
    \toprule
    #4
    \otoprule
    #5
    \bottomrule
   \end{tabular}
   \caption{#1}
   \label{tabla:#3}
  \end{center}
 \end{table}
}

\newcommand{\tablaApaisadaSmall}[5]{%
\begin{landscape}
  \begin{table}
   \begin{center}
    \rowcolors {2}{gray!35}{}
    \begin{tabular}{#2}
     \toprule
     #4
     \otoprule
     #5
     \bottomrule
    \end{tabular}
    \caption{#1}
    \label{tabla:#3}
   \end{center}
  \end{table}
\end{landscape}
}

%
% Nuevo comando para tablas grandes con cabecera y filas alternas coloreadas en gris.
\newcommand{\tabla}[6]{%
  \begin{center}
    \tablefirsthead{
      \toprule
      #5
      \otoprule
    }
    \tablehead{
      \multicolumn{#3}{l}{\small\slshape continúa desde la página anterior}\\
      \toprule
      #5
      \otoprule
    }
    \tabletail{
      \hline
      \multicolumn{#3}{r}{\small\slshape Continúa en la página siguiente}\\
    }
    \tablelasttail{
      \hline
    }
    \bottomcaption{#1}
    \rowcolors {2}{gray!35}{}
    \begin{xtabular}{#2}
      #6
      \bottomrule
    \end{xtabular}
    \label{tabla:#4}
  \end{center}
}

%
% Nuevo comando para tablas grandes con cabecera.
\newcommand{\tablaSinColores}[6]{%
  \begin{center}
    \tablefirsthead{
      \toprule
      #5
      \otoprule
    }
    \tablehead{
      \multicolumn{#3}{l}{\small\slshape continúa desde la página anterior}\\
      \toprule
      #5
      \otoprule
    }
    \tabletail{
      \hline
      \multicolumn{#3}{r}{\small\slshape continúa en la página siguiente}\\
    }
    \tablelasttail{
      \hline
    }
    \bottomcaption{#1}
    \begin{xtabular}{#2}
      #6
      \bottomrule
    \end{xtabular}
    \label{tabla:#4}
  \end{center}
}

%
% Nuevo comando para tablas grandes sin cabecera.
\newcommand{\tablaSinCabecera}[5]{%
  \begin{center}
    \tablefirsthead{
      \toprule
    }
    \tablehead{
      \multicolumn{#3}{l}{\small\sl continúa desde la página anterior}\\
      \hline
    }
    \tabletail{
      \hline
      \multicolumn{#3}{r}{\small\sl continúa en la página siguiente}\\
    }
    \tablelasttail{
      \hline
    }
    \bottomcaption{#1}
  \begin{xtabular}{#2}
    #5
   \bottomrule
  \end{xtabular}
  \label{tabla:#4}
  \end{center}
}



\definecolor{cgoLight}{HTML}{EEEEEE}
\definecolor{cgoExtralight}{HTML}{FFFFFF}

%
% Nuevo comando para tablas grandes sin cabecera.
\newcommand{\tablaSinCabeceraConBandas}[5]{%
  \begin{center}
    \tablefirsthead{
      \toprule
    }
    \tablehead{
      \multicolumn{#3}{l}{\small\sl continúa desde la página anterior}\\
      \hline
    }
    \tabletail{
      \hline
      \multicolumn{#3}{r}{\small\sl continúa en la página siguiente}\\
    }
    \tablelasttail{
      \hline
    }
    \bottomcaption{#1}
    \rowcolors[]{1}{cgoExtralight}{cgoLight}

  \begin{xtabular}{#2}
    #5
   \bottomrule
  \end{xtabular}
  \label{tabla:#4}
  \end{center}
}



\graphicspath{ {./img/} }

% Capítulos
\chapterstyle{bianchi}
\newcommand{\capitulo}[2]{
	\setcounter{chapter}{#1}
	\setcounter{section}{0}
	\setcounter{figure}{0}
	\setcounter{table}{0}
	\chapter*{#2}
	\addcontentsline{toc}{chapter}{#2}
	\markboth{#2}{#2}
}

% Apéndices
\renewcommand{\appendixname}{Apéndice}
\renewcommand*\cftappendixname{\appendixname}

\newcommand{\apendice}[1]{
	%\renewcommand{\thechapter}{A}
	\chapter{#1}
}

\renewcommand*\cftappendixname{\appendixname\ }

% Formato de portada
\makeatletter
\usepackage{xcolor}
\newcommand{\tutor}[1]{\def\@tutor{#1}}
\newcommand{\course}[1]{\def\@course{#1}}
\definecolor{cpardoBox}{HTML}{E6E6FF}
\def\maketitle{
  \null
  \thispagestyle{empty}
  % Cabecera ----------------
\noindent\includegraphics[width=\textwidth]{cabecera}\vspace{1cm}%
  \vfill
  % Título proyecto y escudo informática ----------------
  \colorbox{cpardoBox}{%
    \begin{minipage}{.8\textwidth}
      \vspace{.5cm}\Large
      \begin{center}
      \textbf{TFG del Grado en Ingeniería Informática}\vspace{.6cm}\\
      \textbf{\LARGE\@title{}}
      \end{center}
      \vspace{.2cm}
    \end{minipage}

  }%
  \hfill\begin{minipage}{.20\textwidth}
    \includegraphics[width=\textwidth]{escudoInfor}
  \end{minipage}
  \vfill
  % Datos de alumno, curso y tutores ------------------
  \begin{center}%
  {%
    \noindent\LARGE
    Presentado por \@author{}\\ 
    en Universidad de Burgos --- \@date{}\\
    Tutor: \@tutor{}\\
  }%
  \end{center}%
  \null
  \cleardoublepage
  }
\makeatother

\newcommand{\nombre}{Gonzalo Burgos de la Hera} %%% cambio de comando

% Datos de portada
\title{Análisis y predicción de datos obtenidos de un AGV}
\author{\nombre}
\tutor{Bruno Baruque Zanón y Jesús Enrique Sierra García}
\date{\today}

\begin{document}

\maketitle


\newpage\null\thispagestyle{empty}\newpage


%%%%%%%%%%%%%%%%%%%%%%%%%%%%%%%%%%%%%%%%%%%%%%%%%%%%%%%%%%%%%%%%%%%%%%%%%%%%%%%%%%%%%%%%
\thispagestyle{empty}


\noindent\includegraphics[width=\textwidth]{cabecera}\vspace{1cm}

\noindent D. Bruno Baruque Zanón, profesor del departamento de Digitalización, área de  Ciencia de la Computación e Inteligencia Artificial, y 
D. Jesús Enrique Sierra García, profesor del departamento de Digitalización, área de Ingeniería de Sistemas y Automática.

\noindent Exponen:

\noindent Que el alumno D. \nombre, con DNI 71312090S, ha realizado el Trabajo final de Grado en Ingeniería Informática titulado Análisis y predicción de datos obtenidos del funcionamiento de un AGV. 

\noindent Y que dicho trabajo ha sido realizado por el alumno bajo la dirección del que suscribe, en virtud de lo cual se autoriza su presentación y defensa.

\begin{center} %\large
En Burgos, {\large \today}
\end{center}

\vfill\vfill\vfill

% Author and supervisor
\begin{minipage}{0.45\textwidth}
\begin{flushleft} %\large
Vº. Bº. del Tutor:\\[2cm]
D. Bruno Baruque Zanón
\end{flushleft}
\end{minipage}
\hfill
\begin{minipage}{0.45\textwidth}
\begin{flushleft} %\large
Vº. Bº. del co-tutor:\\[2cm]
D. Jesús Enrique Sierra García
\end{flushleft}
\end{minipage}
\hfill

\vfill

% para casos con solo un tutor comentar lo anterior
% y descomentar lo siguiente
%Vº. Bº. del Tutor:\\[2cm]
%D. nombre tutor


\newpage\null\thispagestyle{empty}\newpage




\frontmatter

% Abstract en castellano
\renewcommand*\abstractname{Resumen}
\begin{abstract}
En cualquier entorno industrial, poder predecir el comportamiento de un sistema de manera precisa ofrece 
una gran ventaja, pues permite ahorrar costes mejorando la productividad.

En este trabajo se propone por tanto el desarrollo de un sistema capaz de 
almacenar y predecir datos de un AGV (Autonomous Guided Vehicle), concretamente la velocidad de sus ruedas. 
Con el fin de conseguir el sistema más óptimo y eficiente posible, se ha realizado un estudio comparativo 
entre diferentes sistemas gestores de bases de datos y de modelos predictivos. Con el fin de obtener la 
mayor modularidad posible, este sistema ha sido desarrollado como un conjunto de microservicios que 
interaccionan entre sí.
\end{abstract}

\renewcommand*\abstractname{Descriptores}
\begin{abstract}
AGV, series temporales, gestores de bases de datos, modelos predictivos, aprendizaje automático, microservicios, \ldots
\end{abstract}

\clearpage

% Abstract en inglés
\renewcommand*\abstractname{Abstract}
\begin{abstract}
In any industrial environment, being able to predict the behaviour of a system in an accurate way offers a 
great advantage, since it allows cost savings by improving productivity.

This work proposes the development of a system capable of storing and predicting data from an AGV (Autonomous Guided Vehicle),
specifically the speed of its wheels. 
In order to achieve the most optimal and efficient system possible, a comparative study has been carried out between different 
database and predictive model management systems.
In order to obtain the greatest possible modularity, this system has been developed as a set of microservices that interact with each other.
\end{abstract}

\renewcommand*\abstractname{Keywords}
\begin{abstract}
AGV, time series, database management systems, predictive models, machine learning, microservices, \ldots
\end{abstract}

\clearpage

% Indices
\tableofcontents

\clearpage

\listoffigures

\clearpage

\listoftables
\clearpage

\mainmatter
\capitulo{1}{Introducción}

% Descripción del contenido del trabajo y del estrucutra de la memoria y del resto de materiales entregados.

Los AGV, Autonomous Guided Vehicles por sus siglas en inglés, son complejos sistemas robóticos, 
capaces de moverse en un entorno concreto, cuyo uso es transportar cargas pesadas en fábricas o 
almacenes, y que están diseñados para mejorar la eficiencia y la productividad en la logística 
y el transporte de materiales. Debido a sus ventajas en seguridad, flexibilidad y velocidad,
esta tecnología se está convirtiendo cada vez más importante \cite{espinosa2020transporte}.

Aunque estos sistemas pueden mejorar la productividad, desajustes en su configuración u otros
errores operacionales pueden producir una reducción de su rendimiento, y, en casos extremos,
causar una detención de la línea de producción. Por este motivo, es necesario extraer información
de los sistemas en marcha para analizar el rendimiento de las máquinas y las aplicaciones logísticas.
Esta información puede usarse para predecir comportamientos futuros del sistema, realizar mantenimiento
predictivo y proveer retroalimentación con el fin de diseñar mejoras continuas de las máquinas. Estas
predicciones pueden ser conseguidas con el uso de algoritmos de análisis de series temporales, que
permitan anticipar futuras condiciones del sistema. \cite{BARUQUE201949}

Algunos de estos datos obtenidos del sistema tienen una baja frecuencia de actualización, como puede
ser la temperatura y voltaje de la batería, pero otros cambian cada pocos milisegundos, como la
corriente eléctrica, la velocidad, la posición del vehículo, errores y estado, etc. Toda esta
información proveída por el AGV debe estar relacionada con el tiempo en el que fue generada, por
lo que puede ser agrupada en series temporales \cite{DBLP:journals/corr/abs-2104-00164}. Cualquier
tipo de base de datos puede usarse para almacenar esta información generada por los AGV, sin embargo,
ya que se trata de series temporales, es preferible utilizar bases de datos para series temporales
para optimizar el rendimiento del sistema.


\capitulo{2}{Objetivos del proyecto}

El objetivo principal de este trabajo es el de desarrollar un sistema dedicado a un entorno profesional, capaz 
de mejorar la eficiencia y reducir costes en procesos industriales.

Este proyecto ha sido diseñado como un microservicio. Las aplicaciones diseñadas de esta manera se dividen 
en elementos más pequeños e independientes entre sí, y que deben comunicarse y coordinarse. De esta 
forma, se consigue que la aplicación sea modular, más facil de escalar, de mantener y de desarrollar.

Al dividir la aplicación en lo que a efectos prácticos son aplicaciones más simples, se consigue un código 
mucho más sencillo, por lo que el diseño del propio código se simplifica. En nuestro caso, se ha utilizado 
un enfoque de programación funcional, en vez de programación orientada a objetos, pues la simplicidad del propio 
código hace que esta última metodología no sea la más apropiada.

Los objetivos pueden quedar resumidos en la siguiente lista:
\begin{itemize}
    \item Desarrollar la aplicación como un microservicio.
    \item Utilizar Docker para agrupar cada servicio en su contenedor independiente.
    \item Utilizar un enfoque de programación funcional.
    \item Realizar un estudio comparativo de sistemas gestores de bases de datos.
    \item Realizar otro estudio comparando modelos de predicción de series temporales.
    \item Hacer uso de Git como herramienta de control de versiones.
    \item Utilizar herramientas de integración continua como CodeClimate.
    \item Aplicar la metodología ágil durante el desarrollo del software.
\end{itemize}
\capitulo{3}{Conceptos teóricos}

Como se ha mencionado en apartados anteriores, los datos recibidos por el AGV se agrupan en series temporales. Conviene por tanto explicar
que son las series temporales, así como los modelos que se utilizarán para intentar predecirlas.


\section{Series temporales}

Una serie temporal es una colección de observaciones obtenidas mediante mediciones repetidas a lo largo del tiempo \cite{influx:timeseries}.

Normalmente, las series temporales presentan patrones que pueden utilizarse para realizar predicciones. Estos patrones son:
\begin{itemize}
    \item \textbf{Tendencia}. La tendencia existe cuando hay un incremento o decremento del valor medido a largo plazo. Esta tendencia
        no tiene por qué ser lineal.
    \item \textbf{Estacionalidad}. El patrón de estacionalidad se presenta cuando una serie temporal se ve afectada por factores estacionales,
        como puede ser el día de la semana. Siempre tiene una frecuencia fija y conocida.
    \item \textbf{Ciclos}. Un ciclo ocurre cuando los datos muestran incrementos o decrementos a una frecuencia no fija.
\end{itemize}

\imagen{pltEncDerecho.pdf}{Ejemplo de serie temporal}{1}

Existen distintos tipos de clasificaciones para las series temporales, según varios puntos de vista \cite{kitagawa2010introduction}, de
las cuales destacan:
\begin{itemize}
    \item \textbf{Continuas o discretas}. Las series temporales continuas son aquellas en las que la información se obtiene, valga la redundancia,
        de forma continua, normalmente por un dispositivo analógico, como podría ser los datos recibidos de un sismógrafo. Por otro lado,
        las series temporales discretas son aquellas en las que la información se obtiene en intervalos de tiempo concretos. Estos intervalos 
        pueden ser equidistantes, o bien ser irregulares. Normalmente, las series temporales medidas por medios digitales son discretas.
        En nuestro caso, las series temporales enviadas por el AGV son discretas espaciadas en intervalos irregulares, pues el AGV manda dicha
        información cada varios milisegundos, de manera irregular.
    \item \textbf{Univariantes o multivariantes}. Aquellas series temporales que tengan solo una observación por cada momento del tiempo son series
        univariantes. Por contra, aquellas en las que se obtengan de manera simultánea mediciones de dos o más fenómenos son multivariantes.
        Esto será importante a la hora de escoger que modelo utilizar para realizar predicciones, pues hay modelos que solo soportan series
        temporales univariantes.
    \item \textbf{Estacionarias o no estacionarias}. Una serie estacionaria \cite{hyndman2018forecasting} es aquella en la que sus propiedades no dependen del momento
        en el que se observan. Por ello, aquellas que presenten tendencias o estacionalidad no son estacionarias. Por otra parte, una serie de 
        ruido blanco es estacionaria: no importa cuándo se observe, debería tener el mismo aspecto en cualquier momento. De manera general, las
        series estacionarias no tendrán patrones predecibles en el largo plazo, por lo que conviene convertir las series estacionarias en no
        estacionarias.
\end{itemize}

\section{Predicción de series temporales}

La predicción de series temporales es una actividad muy importante en muchos sectores: predicción 
de datos financieros, predicciones del clima, etc. Debido a esto, existe una gran cantidad de 
modelos para realizar dichas predicciones. Hay que tener en cuenta sin embargo que predecir 
datos futuros es una tarea especialmente complicada, y no siempre se obtiene una gran precisión.

Los siguientes modelos se tendrán en cuenta en este trabajo:
\begin{itemize}
    \item \textbf{ARIMA.} Este modelo (Autoregressive Integrated Average) \cite{hyndman2018forecasting} es un enfoque estadístico utilizado para 
        el análisis y pronostico de series temporales. Es una combinación de tres componentes principales:
        \begin{itemize}
            \item Componente autorregresivo (AR): este componente utiliza la información de valores pasados de la
                serie temporal. Se basa en la idea de que los valores pasados tienen influencia en el futuro. Este 
                modelo indica cuantos valores pasados se utilizan en la predicción.
            \item Componente de media móvil (MA): tiene en cuenta el error residual de las predicciones anteriores 
                para mejorar la precisión, haciendo una media de los errores pasados para predecir los futuros. 
                Indica cuantos errores pasados se tienen en cuenta.
            \item Componente de Integración (I): se refiere al proceso de diferenciación de la serie temporal para 
                hacerla estacionaria. El orden de Integración indica cuántas veces se diferencia la serie temporal.
        \end{itemize}
        La combinación de estos tres componentes forman el modelo ARIMA(p, d, q), donde ``p'' representa el orden 
        del componente autorregresivo, ``d'' es el orden del componente de integración y ``q'' es el orden del componente 
        de media móvil.
    \item \textbf{TCN.} Una Red neuronal Convolucional Temporal (Convolutional Temporal Network en inglés) \cite{DBLP:journals/corr/abs-1803-01271} es un tipo de 
        red neuronal utilizada para analizar series temporales. Las TCN tienen en cuenta la estructura temporal de los datos 
        y aplican operaciones convolucionales para capturar patrones. Una convolución \cite{matlab:convolucion} es un operador matemático que transforma 
        dos funciones en una nueva. Un ejemplo de convolución es la media móvil, o un filtro de aumento de nitidez para imágenes.

        TCN utiliza capas convolucionales de una dimensión para aprender características de la serie temporal. Estas 
        capas son aplicadas sobre ventanas deslizantes de la secuencia para extraer características en diferentes puntos 
        de tiempo.
    \item \textbf{N-HiTS.} Este modelo (Neural Hierarchical interpolation for Time Series) es una extensión del modelo 
        N-BEATS, mejorando su rendimiento y velocidad de entrenamiento \cite{DBLP:journals/corr/abs-2201-12886}. N-BEATS \cite{Oreshkin2020N-BEATS:}
        está formado por dos componentes: stack y bloque. Un bloque está formado por una red multicapa que predice 
        valores futuros y pasados. Estos bloques se organizan en pilas (stacks), que agregan las predicciones y errores 
        residuales. 
    \item \textbf{Transformer Model.} El Modelo Transformador \cite{DBLP:journals/corr/VaswaniSPUJGKP17} es una red 
        neuronal que aprende el contexto de la información. Este modelo usa una arquitectura codificador-decodificador,
        en la que el codificador procesa la entrada de forma iterativa, y el decodificador hace lo mismo con la salida del 
        codificador.
    \end{itemize}

Estos modelos serán los utilizados para la comparación en el desarrollo del proyecto. Para ello, se utilizan 
las siguientes métricas:
\begin{itemize}
    \item \textbf{MAE}. Siglas de Mean Absolute Error, o Error Absoluto Medio en español. Mide el error medio
        de una predicción sin tener en cuenta la dirección de dicho error.
    \item \textbf{MASE}. El MASE \cite{ibm:docs/metrics} (Mean Absolte Scaled Error), o error absoluto escalado medio, mide como de bueno es 
        el modelo comparado con un modelo ``ingenuo'' (modelo que predice un valor como su valor previo). Un valor 
        por encima de 1 significa que nuestro modelo es peor que dicho modelo ingenuo.
    \item \textbf{DTW}. Siglas de Dynamic Time Warping, es utilizado para medir la similitud entre dos series temporales. Por ejemplo, una predicción 
        con forma de ecuación lineal puede tener el mismo MAE que otra predicción más irregular, pero esta segunda 
        puede parecerse más a los datos reales.
\end{itemize}
\capitulo{4}{Técnicas y herramientas}

Esta parte de la memoria tiene como objetivo presentar las técnicas metodológicas y las herramientas de desarrollo que se han utilizado para llevar a cabo el proyecto. Si se han estudiado diferentes alternativas de metodologías, herramientas, bibliotecas se puede hacer un resumen de los aspectos más destacados de cada alternativa, incluyendo comparativas entre las distintas opciones y una justificación de las elecciones realizadas. 
No se pretende que este apartado se convierta en un capítulo de un libro dedicado a cada una de las alternativas, sino comentar los aspectos más destacados de cada opción, con un repaso somero a los fundamentos esenciales y referencias bibliográficas para que el lector pueda ampliar su conocimiento sobre el tema.


\tablaSmallSinColores{Comparación gestores de Bases de Datos}{l c c c c}{compbd}
{\multicolumn{1}{l}{Herramientas} & SQLite & VoltDB & InfluxDB & MongoDB \\}
{
% Tipo & SQL & NewSQL & \multirow{2}{l}{} & NoSQL \\
\multirow{2}*{Tipo} & \multirow{2}*{SQL} & \multirow{2}*{NewSQL} & NoSQL & NoSQL \\
                    &                    &                       & TSDB & DBDB \\
Relacional & Sí & Sí & No & No \\
\multirow{2}*{}Almacenamiento & \multirow{2}*{No} & \multirow{2}*{Sí} & \multirow{2}*{No} & \multirow{2}*{No} \\
en memoria & & & & \ \\
Soporte para Linux & Sí & Sí & Sí & Sí \\
}
\capitulo{5}{Aspectos relevantes del desarrollo del proyecto}

% Este apartado pretende recoger los aspectos más interesantes del desarrollo del proyecto, comentados por los autores del mismo.
% Debe incluir desde la exposición del ciclo de vida utilizado, hasta los detalles de mayor relevancia de las fases de análisis, diseño e implementación.
% Se busca que no sea una mera operación de copiar y pegar diagramas y extractos del código fuente, sino que realmente se justifiquen los caminos de solución que se han tomado, especialmente aquellos que no sean triviales.
% Puede ser el lugar más adecuado para documentar los aspectos más interesantes del diseño y de la implementación, con un mayor hincapié en aspectos tales como el tipo de arquitectura elegido, los índices de las tablas de la base de datos, normalización y desnormalización, distribución en ficheros3, reglas de negocio dentro de las bases de datos (EDVHV GH GDWRV DFWLYDV), aspectos de desarrollo relacionados con el WWW...
% Este apartado, debe convertirse en el resumen de la experiencia práctica del proyecto, y por sí mismo justifica que la memoria se convierta en un documento útil, fuente de referencia para los autores, los tutores y futuros alumnos.

El desarrollo de este proyecto puede decirse que ha estado dividido principalmente en dos partes. La primera
fase estuvo relacionada con el diseño de la arquitectura, así como la elección del sistema gestor de bases de
datos. En la segunda fase, se diseñó todo lo relativo a la predicción de datos y a la elección de las
herramientas necesarias para dicha tarea.

\section{Primera fase}

Como ha sido mencionado en el párrafo anterior, en la primera fase se llevó a cabo el diseño de la
arquitectura del sistema, así como el desarrollo de los servicios que forman dicho sistema. Además,
se ha realizado una comparativa de diferentes sistemas gestores de bases de datos con el fin de elegir
el que mejor se adapte a nuestros requisitos.

\subsection*{Diseño de la arquitectura}

La arquitectura del sistema estudiado se representa en la figura \ref*{fig:Arquitectura}. El sistema
está formado por los siguientes sistemas software:
\begin{description}
    \item [AGV Coordinator] Es un servicio encargado de recibir la información enviada por los AGV a
        través de una conexión 5G/Wifi. Esta información está codificada como una cadena de bytes, que
        es decodificada y transformada a formato JSON. Estos mensajes son enviados al servicio 
        ``Receiver''. Este servicio no ha sido desarrollado como parte de este trabajo.
    \item [Simulator] Es, como su nombre en inglés indica, un servicio encargado de simular un AGV
        en caso de no disponer de AGV reales y sea necesario realizar pruebas.
    \item [Receiver] Este servicio recibe mensajes a través del protocolo UDP bien del ``AGV Coordinator''
        o bien del simulador, y se encarga de insertar dichos datos en la base de datos.
    \item [Database] Como su nombre indica, este servicio será la base de datos encargada de almacenar
        todos los datos recibidos.
\end{description}

\imagen{Arquitectura}{Diseño de la arquitectura}{1}

\subsection*{Comparativa de Sistemas Gestores de Bases de Datos}

Lo primero a tener en cuenta a la hora de elegir un gestor de bases de datos es cuáles son nuestros
requisitos relacionados con el almacenamiento de los datos. Como hemos comentado anteriormente, el tiempo
juega un papel importante en los mensajes que enviamos, pues es importante saber cuando el valor de
una variable, como puede ser el nivel de la batería, fue reportado. Algunas de estas variables son
enviadas cada 10 o 20 milisegundos, por lo que una alta precisión temporal es necesaria.

\begin{table}[H]
    \begin{tabularx}{\textwidth}{l X}
        \toprule
        Requisito                 & Descripción \\
        \otoprule
        RF1:                & El sistema gestor de bases de datos debe soportar entradas con timestamp \\
        \rowcolor{gray!35}
        RF2:                & Los datos tienen que poder insertarse en cualquier momento \\
        RF3:                & El timestamp de las entradas de la base de datos ha de corresponder con el momento en el que dichos datos fueron creados en el AGV \\
        \rowcolor{gray!35}
        RF4:                & El sistema gestor de bases de datos debe poder soportar entradas con una precisión temporal en el rango de los milisegundos \\
        RNF1:                & El sistema gestor de base de datos debe ser de código abierto \\
        \rowcolor{gray!35}
        RNF2:                & EL sistema gestor de base de datos debe tener un buen soporte de Linux y Python \\
        \bottomrule
    \end{tabularx}
    \caption{Requisitos Funcionales (RF) y no Funcionales (RNF) de la base de datos}
    \label{tabla:req}
\end{table}

\subsubsection*{Metodología de la comparación}

\paragraph*{Sistemas de gestión de bases de datos bajo estudio}

Lo primero a tener en cuenta es el modelo de la base de datos que se va a utilizar, ya que tendrá una gran importancia
a la hora de decidir qué sistema de gestión de bases de datos se va a utilizar. Los siguientes modelos \cite{10.5555/3364297} han sido
tenidos en cuenta: 

\begin{enumerate}
    \item Bases de datos relacionales: en estos sistemas, la información se almacena en relaciones. Una relación,
        definidas también como tablas, es una colección de tuplas, o filas. Las relaciones se definen por su nombre
        y un número fijo de atributos, o columnas, con tipos de datos fijos. Estos sistemas respetan las propiedades
        ACID (Atomicity, Consistency, Isolation y Durability), y tienen operaciones básicas definidas, como la selección,
        proyección y unión.
    \item Bases de datos de documentos: la principal característica de estas bases de datos es la organización de los datos
        de forma libre, sin seguir ningún esquema. Esto significa, por contrario que en las bases de datos relacionales,
        las entradas no poseen una estructura uniforme, las columnas pueden tener más de un valor e incluso pueden almacenar
        estructuras anidadas.
    \item Bases de datos de Clave-valor: son las bases de datos más simples y solo almacenan pares clave-valor. Normalmente
        no son factibles para aplicaciones complejas, pero normalmente presentan un gran rendimiento debido a su
        simplicidad.
    \item Motores de búsqueda: su uso principal es la búsqueda de datos, y son típicamente NoSQL, es decir, que no siguen
        un modelo relacional.
    \item Bases de datos de series temporales: estas bases de datos están optimizadas para almacenar series temporales \cite{influx:timeseries}.
        Aunque son típicamente NoSQL, bases de datos de series temporales relacionales existen.
\end{enumerate}

Cualquiera de estos modelos permiten el manejo de información de series temporales, como la información que recibimos
de los AGV. Sin embargo, las bases de datos de series temporales están especializadas en este tipo de trabajos, por lo que
las hace perfectas para nuestras necesidades.

Como selección inicial, se han escogido los cinco sistemas gestores de bases de datos de series temporales más temporales
según el ranking DB-Engines \cite{dbengines:rankingTSDBMS}. Este ranking es utilizado en otros estudios, como \cite{10.1007/978-3-030-50426-7_28}.
Los sistemas gestores de bases de datos elegidos para comparar son:

\begin{description}
    \item[InfluxDB 2.6.1] Un sistema gestor de bases de datos de series temporales desarrollado por InfluxData. Su
        uso principal es el almacenamiento y obtención de series temporales, creadas en operaciones de monitoreo de IoT,
        información de sensores, etc.
    \item[kdb+ 4.0] Una base de datos de series temporales relacional desarrollado por KX, usado principalmente en
        negociación bursátil de alta frecuencia, para almacenar y para procesar datos a una alta velocidad.
    \item[Prometheus 2.43.0] Una base de datos de series temporales usada para el monitoreo de eventos y alarmas, que
        utiliza un modelo HTTP.
    \item[Graphite 1.1.10] Una herramienta que almacena, monitoriza y grafica series temporales numéricas.   
    \item[TimescaleDB 2.10.1] Una base de datos de código abierto, desarrollada como complemento a PostgreSQL con el
        fin de mejorar el rendimiento y análisis para series temporales.
\end{description}

Aunque este ranking solo mide la popularidad y ordena los sistemas en función de atributos sociales, es especialmente
útil para soluciones de código abierto, pues esto generalmente significa que está soportada por una comunidad activa
con muchos colaboradores involucrados en añadir nueva funcionalidad y corregir errores.

\paragraph*{Procedimiento de la comparación}
La comparación y el filtrado de los modelos seleccionados ha sido realizados en tres pasos secuenciales:
\begin{enumerate}
    \item Información general, soporte software y apoyo de la comunidad.
    \item Modelo de datos y especificaciones técnicas.
    \item Prueba de rendimiento.
\end{enumerate}

En los primeros dos pasos, los sistemas comparados que no cumplan los requisitos especificados en la tabla \ref{tabla:req}
han sido descartados. Después, con los sistemas restantes, se ha realizado una prueba de rendimiento con el fin de
tomar la decisión final. A su vez, esta prueba se divide en otras dos pruebas.

\imagen{DiagramaTests1}{Procedimiento de la prueba de inserción}{1}
\imagen{DiagramaTests2}{Procedimiento de la prueba de latencia}{1}

El primer test, (Figura \ref{fig:DiagramaTests1}) mide el uso de CPU y RAM del sistema cuando se insertan datos de forma
masiva, así como el tiempo tomado y el número de inserciones por segundo realizadas. En total, 300.000 entradas son enviadas,
formadas por los siguientes campos: timestamp, id del vehículo, batería, velocidad, posición en la coordenada x, posición
en la coordenada y, temperatura y voltaje. Para realizar la prueba de manera más realista, los datos se insertan en la
base de datos en tandas de 5.000. Para ello se almacenan primero en un buffer controlado por el servicio ``Receiver'', ya
que de esta manera se obtiene un mejor rendimiento que si se insertasen de uno en uno. Para medir el uso de CPU y de RAM
de la misma forma con todos los diferentes gestores, las medidas son tomadas de lo que reporta el estado del contenedor
de Docker en el que se ejecutan dichos sistemas.

En el segundo test (Figura \ref*{fig:DiagramaTests2}) se mide la latencia de inserción. Esto es, el tiempo que tarda
en estar disponible una entrada después de su inserción. Para esto, un script de Python, formado por dos hilos, ha sido creado.
El primero de esos hilos realiza la inserción en la base de datos y el otro intenta obtener dicha entrada en un bucle.
En el momento en el que dicha entrada se obtiene, se anota dicho tiempo y se resta del momento en le que se insertó
la entrada. Este test se realiza 200 veces y se hace una media con los resultados.

Cada test se realiza cinco veces para reducir la variabilidad, tomando la media de dichas ejecuciones como valor final.

\paragraph*{Métricas de la comparación} 
A continuación se detallan las métricas utilizadas para realizar la comparación.

\textbf{Información general, soporte de software y comunidad}

Las métricas de información general analizan:
\begin{itemize}
    \item Organización: que organización o compañía es responsable del desarrollo y mantenimiento del sistema.
    \item Año de lanzamiento: en que año se lanzó inicialmente.
    \item Última versión: en que año se lanzó la última versión.
    \item Licencia: que tipo de licencia tiene el software: código abierto (OSS), o licencia comercial.
\end{itemize}
El indicador de rendimiento del soporte de software analiza:
\begin{itemize}
    \item Sistema Operativo: qué sistemas operativos se soportan.
    \item Soporte para Python: como el sistema está implementado en Python, nos interesa que el sistema tenga
        buen soporte de este lenguaje.
    \item Lenguaje de consultas: que lenguaje de consultas soporta el sistema.
    \item Plugins para aprendizaje automático: si el sistema soporta plugins que simplifiquen la predicción de nuevos
        datos.
\end{itemize}
La comparación del soporte de la comunidad contiene:
\begin{itemize}
    \item Número de estrellas del repositorio de GitHub: como forma de medir su popularidad.
    \item Pull requests: número de pull requests enviadas en el último mes.
    \item Pull requests aceptadas: número de pull requests aceptadas en el último mes.
    \item Issues: número de issues creados en el último mes.
    \item Issues cerrados: número de issues cerrados en el último mes.
\end{itemize}
Estos cuatro últimos campos se usarán para comparar que comunidad es más activa.

\textbf{Modelo de datos e información técnica}

El indicador de rendimiento del modelo de datos compara:
\begin{itemize}
    \item Modelo de datos: que modelo concreto implementa cada sistema.
    \item Esquema: un esquema puede verse como una plantilla que define como se almacena la información. Este campo
        compara si la organización de los datos es estricta (esquema fijo) o no (esquema libre).
    \item Índices secundarios: si el sistema soporta índices secundarios para un mejor rendimiento de consultas o no.
    \item Precisión temporal: unidad mínima de tiempo que puede tener una entrada.
\end{itemize}

El indicador de rendimiento de información técnica se compone de los siguientes campos:
\begin{itemize}
    \item Scripts de servidor: si el sistema es capaz de ejecutar scripts en el servidor o no.
    \item Método de partición: si se soportan o no métodos de partición para una mayor escalabilidad.
    \item Replicación: que métodos de replicación soporta.
    \item Consistencia: si la información escrita es consistente o no.
    \item Conformidad con ACID: si el sistema sigue los principios ACID o no.
    \item Concurrencia: si el sistema soporta accesos concurrentes o no.
    \item Durabilidad: si la información es persistente, incluso si falla.
    \item Método de inserción: si la información se introduce mediante una consulta de inserción o extrayendo los
        datos de un endpoint de forma periódica.
\end{itemize}

\textbf{Análisis de rendimiento}

Por último, el análisis de rendimiento compara:
\begin{itemize}
    \item Tiempo de inserción: tiempo que se tarda en hacer la prueba de inserción en segundos.
    \item Tasa de transferencia: número de inserciones por segundo.
    \item Uso de la CPU: uso de la CPU del contenedor de Docker en el que se ejecuta base de datos durante 
        la primera prueba.
    \item Uso de RAM: uso de RAM del contenedor de Docker en el que se ejecuta la base de datos durante la
        primera prueba.
    \item Latencia: tiempo que tarda en ejecutarse la segunda prueba en milisegundos.
\end{itemize}

\subsubsection*{Experimentos y resultados}

\begin{table}[H]
    \begin{center}
        \begin{adjustbox}{max width=\textwidth}
            \rowcolors{2}{gray!35}{}
            \begin{tabular}{l c c c c}
                \toprule
                Systems & Organization & Launch year & Latest version & License \\
                \otoprule
                InfluxDB    & InfluxData & 2013 & 2023 & OSS \\
                db+        & Kx Systems & 2000 & 2020 & Comercial \\
                Prometheus  & -          & 2015 & 2023 & OSS \\
                Graphite    & -          & 2006 & 2022 & OSS \\
                TimescaleDB & Timescale  & 2017 & 2023 & OSS \\
                \bottomrule
            \end{tabular}
        \end{adjustbox}
        \caption{Comparativa de información general}
        \label{tabla:gisgbd}
    \end{center}
\end{table}

\paragraph*{Información general (Tabla \ref*{tabla:gisgbd})} Solo software de código abierto será considerado en este
trabajo. Aunque kdb+ tiene una versión de 32 bits, no se usará y no volverá a aparecer en las siguientes comparaciones.
Por esta razón también, solo las características de la ``Comunity Edition'' de InfluxDB serán utilizadas en dichas
comparaciones, y características de la ``Enterprise Edition'', que no es de códigp abierto, no se tendrán en cuenta.

\begin{table}[H]
    \begin{center}
        \begin{adjustbox}{max width=\textwidth}
            \begin{tabular}{l c c c c}
                \toprule
                System & OS & Python & Query language & ML Plugins\\
                \otoprule
                & Linux &                       &  \\
                \multirow{-2}{*}{InfluxDB} & OS x  & \multirow{-2}{*}{Sí} & \multirow{-2}{*}{Flux and InfluxQL} & \multirow{-2}{*}{Loud ML} \\
                \rowcolor{gray!35}
                                            & Linux   &                        & & \\
                \rowcolor{gray!35}
                \multirow{-2}{*}{Prometheus} & Windows & \multirow{-2}{*}{Sí}  & \multirow{-2}{*}{PromQL} & \multirow{-2}{*}{No}\\
                                        & Linux &                       &  & \\
                \multirow{-2}{*}{Graphite} & Unix  & \multirow{-2}{*}{Sí}  & \multirow{-2}{*}{No} & \multirow{-2}{*}{No} \\
                \rowcolor{gray!35}
                                            & Linux   &                             & & \\
                \rowcolor{gray!35}
                                            & OS X    &                             & & \\
                \rowcolor{gray!35}
                \multirow{-3}{*}{TimescaleDB} & Windows & \multirow{-3}{*}{Sí} & \multirow{-3}{*}{SQL} & \multirow{-3}{*}{No} \\
                \bottomrule
            \end{tabular}
        \end{adjustbox}
        \caption{Soporte software}
        \label{tabla:sssgbd}
    \end{center}
\end{table}

\paragraph*{Soporte software (Tabla \ref*{tabla:sssgbd})} Todos los sistemas soportan Linux y Python, el sistema operativo 
y lenguaje utilizados. Solo Graphite no tiene un lenguaje de consultas definido, aunque se pueden realizar utilizando lo que 
llaman Funciones \cite{graphite-functions}. Sólo InfluxDB soporta plugins para aprendizaje automático. Otros sistemas como
TimescaleDB proveen documentación para realizarlo de forma externa en lenguajes como Python o R \cite{timescale-forecasting}.

\begin{table}[H]
    \begin{center}
        \begin{adjustbox}{max width=\textwidth}
            \rowcolors{2}{gray!35}{}
            \begin{tabular}{l c c c c c c}
                \toprule
                Sistemas & Estrellas GitHub & Pull requests & Pull requests aceptadas & Issues & Issues cerrados \\
                \otoprule
                InfluxDB    & 25.2k & 26 & 22 & 37 & 13 \\
                Prometheus  & 47.4k & 75 & 53 & 42 & 20\\
                Graphite & 5.6k & 0 & 0 & 0 & 0 \\
                TimescaleDB & 14.7k & 105 & 83 & 67 & 46 \\
                \bottomrule
            \end{tabular}
        \end{adjustbox}
        \caption{Soporte de la comunidad}
        \label{tabla:cssgbd}
    \end{center}
\end{table}

\paragraph*{Soporte de la comunidad (Tabla \ref*{tabla:cssgbd})} Como muestra la tabla, todos los proyectos son muy
activos a excepción de Graphite.

\begin{table}[H]
    \begin{center}
        \begin{adjustbox}{max width=\textwidth}
            \begin{tabular}{l c c c c c}
                \toprule
                \multirow{2}{*}{Sistema} & \multirow{2}{*}{Modelo} & \multirow{2}{*}{Esquema} & \multirow{2}{*}{Tipado} & Índice & Precisión\\
                &&&& secundario & temporal \\
                \otoprule
                &&& Numéricos && \\
                \multirow{-2}{*}{InfluxDB}    & \multirow{-2}{*}{Multidimensional} & \multirow{-2}{*}{Libre} & y strings & \multirow{-2}{*}{No} & \multirow{-2}{*}{Nanosegundos} \\
                \rowcolor{gray!35}
                Prometheus  & Multidimensional & Sí & Numéricos & No & Milisegundos \\
                Graphite    & Key-Value & Sí & Numéricos & No & Segundos \\
                \rowcolor{gray!35}
                TimescaleDB & Relacional & Sí & Tipos SQL & Sí & Nanosegundos \\
                \bottomrule
            \end{tabular}
        \end{adjustbox}
        \caption{Comparativa del modelo de datos}
        \label{tabla:dmsgbd}
    \end{center}
\end{table}

\paragraph*{Comparación del modelo de datos (Tabla \ref*{tabla:dmsgbd})} Tanto InfluxDB como Prometheus utilizan un modelo 
multidimensional. Este modelo puede verse como un modelo clave-valor multidimensional: las entradas de datos están formados 
por un campo que describe la información almacenada (``nombre de la métrica'' para Prometheus y ``medida'' para InfluxDB) 
y un set de pares clave-valor asociados con un timestamp. La principal diferencia es que las entradas en el modelo de InfluxDB 
están formados por la medida, un set de etiquetas y un set de valores, en vez de solo un set de pares clave-valor. 
Estas etiquetas guardan metadatos en forma de cadenas de caracteres, son opcionales y están indexados, mientras que el set de 
valores guardan la información, no están indexados y están asociados con un timestamp.

Graphite agrega los datos de manera automática en ventanas de un segundo o más. Este comportamiento no es el deseado 
para nuestras necesidades, ya que se requiere almacenar todos los datos enviados.

\begin{table}[H]
    \begin{center}
        \begin{adjustbox}{max width=\textwidth}
            \begin{tabular}{l c c c c}
                \toprule
                Sistema & InfluxDB & Prometheus & Graphite & TimescaleDB \\
                \otoprule
                Scripts del servidor & No & No & No & Sí \\
                \rowcolor{gray!35}
                Particionamiento & No & Sharding & No & Sí \\
                Replicación & No & Sí & No & Sí \\
                \rowcolor{gray!35}
                Consistencia & Eventual & No & No & Innmediata \\
                ACID& No & No & No & Sí \\
                \rowcolor{gray!35}
                Concurrencia & Sí & Sí & Sí & Sí \\
                Durabilidad & Sí & Sí & Sí & Sí \\
                \rowcolor{gray!35}
                Permisos                   & Permisos vía    &                      &                      & Derechos \\
                \rowcolor{gray!35}
                de usuario & cuentas & \multirow{-2}{*}{No} & \multirow{-2}{*}{No} & estandar SQL \\
                Método inserción & Push & Pull & Push & Push \\
                \bottomrule
            \end{tabular}
        \end{adjustbox}
        \caption{Comparativa de información técnica}
        \label{tabla:tisgbd}
    \end{center}
\end{table}

\paragraph*{Comparativa información técnica (Tabla \ref*{tabla:tisgbd})} En InfluxDB, la consistencia es eventual. Según
la documentación, se prioriza el rendimiento de lectura y escritura antes que una fuerte consistencia. Se asegura, sin 
embargo, que la información es eventualmente consistente \cite{influx:consistency}.

En bases de datos típicas, la información se inserta a través de algún tipo de consulta desde fuera. Por otro lado, 
Prometheus escucha a un endpoint en el que se publican los datos y se obtienen en intervalos fijos de tiempo. Esto 
significa que la inserción solo ocurrirá cuando Prometheus escuche a dicho endpoint, por lo que la información no se 
puede insertar en cualquier momento. Un método más típico existe, pero no es recomendado y no es posible especificar 
timestamps \cite{prom:pushgateway}.

Solo InfluxDB y TimescaleDB cumplen todos los requisitos, ya que Graphite agrega los datos en ventanas de 1 segundo,
haciendo imposible obtener datos de un momento concreto, y la forma de inserción de Prometheus le hace incompatible 
con nuestras necesidades, ya que es necesario poder insertar datos en cualquier momento. Por esto, solo estos dos
sistemas se compararán en la prueba de rendimiento.

\paragraph*{Análisis del rendimiento} La prueba se realizó con un procesador AMD Ryzen 5 3600 y 32 GB de RAM. Ya que 
este modelo de CPU tiene 12 hilos, el uso de la CPU puede ser tan alto como 1200\% (Uso CPU = Hilos * 100). Los resultados
de esta prueba se muestran en la tabla \ref*{tabla:ptsgbd}.

\tablaSmallFija{Resultados de la prueba de rendimiento}{l c c}{ptsgbd}{
System & InfluxDB & TimescaleDB\\
}{
    Tiempo inserción (s) & 24.13 & \textbf{1.16} \\
    Tasa inserción (I/s) & 12432.66 & \textbf{258620.69} \\
    Uso CPU (\%) & \textbf{15.05} & 55.32 \\
    Uso RAM (MB) & \textbf{219.85} & 373.73 \\
    Latencia (ms) & 3.37 & \textbf{0.22} \\
}

Como se puede observar, InfluxDB es más lento en todos las pruebas que TimescaleDB, pero este último utiliza más 
recursos del sistema. Si en un futuro es necesario escalar InfluxDB puede ser mejor opción, ya que un menor uso de 
recursos suele significar un menor coste. Sin embargo, TimescaleDB es más flexible, ya que al estar basado en PostgreSQL 
tiene todas sus características. Cualquiera de estos dos sistemas puede ser perfectamente usado según los requisitos marcados.

\subsubsection*{Elección}

Al final, el sistema gestor de bases de datos escogido ha sido InfluxDB. Aunque sea más lento que TimescaleDB, tiene
una velocidad lo suficientemente buena, y al consumir menos recursos la hace una opción más barata en caso de que 
esta solución se aplique comercialmente. 
\imagen{interfaz}{Interfaz Chronograf}{1}
Otro motivo de peso para elegir InfluxDB es que viene por defecto con una interfaz web llamada Chronograf 
(Figura \ref*{fig:interfaz}) en la que se puede manejar la base de datos, crear gráficas, establecer alarmas, etc.


\subsection*{Desarrollo de los servicios}

Una vez diseñada la arquitectura y el sistema gestor de bases de datos escogido, se procedió a desarrollar los
sistemas definidos.

\subsubsection*{Simulator}
Inicialmente, no disponía de datos reales del AGV, por lo que la primera versión (Figura \ref*{fig:v1sim}) simplemente generaba datos aleatorios
con campos aleatorios, y los enviaba al nodo ``Receiver'' por UDP utilizando el puerto 5004.

Después, se intentó desarrollar un simulador capaz de, valga la redundancia, simular el comportamiento de un AGV. Sin
embargo, una vez dispuse de datos reales del AGV, esta idea se descartó, pues simular dicha información de forma
precisa iba a ser demasiado complejo, y se escapa del objetivo de este proyecto. Por tanto, se decidió simular el
comportamiento del AGV leyendo los datos de un CSV (Figura \ref*{fig:v2sim}) obtenido a partir de uno real.

\begin{figure}
    \centering
    \begin{subfigure}[b]{0.45\textwidth}
        \centering
        \includegraphics*[width=0.5\textwidth]{v1sim}
        \caption{Generación aleatoria}
        \label{fig:v1sim}
    \end{subfigure}
    \hfill
    \begin{subfigure}[b]{0.45\textwidth}
        \centering
        \includegraphics*[width=0.9\textwidth]{v2sim}
        \caption{Generación por CSV}
        \label{fig:v2sim}
    \end{subfigure}
    \begin{subfigure}[b]{0.7\textwidth}
        \centering
        \includegraphics*[width=\textwidth]{sim}
        \caption{Versión final}
        \label{fig:sim}
    \end{subfigure}
    \caption{Diagramas de flujo del simulador}
    \label{fig:diagsim}
\end{figure}

Por último, en la versión final, se unificaron los dos procedimientos, de forma que el comportamiento del simulador
se decide según lo especificado en un archivo de configuración.

\subsubsection*{Receiver}

El funcionamiento del nodo ``Reciever'' es muy simple (Figura \ref*{fig:recv}): escucha el puerto UDP 5004, y cuando recibe información, la inserta 
en la base de datos. Inicialmente, el servicio intentará conectarse a la base de datos. Si esta conexión falla un número
determinado de veces, el servicio fallará informando de que la conexión no ha podido realizarse.

\imagen{recv}{Diagrama de flujo del servicio ``Reciever''}{0.5}

\section{Segunda fase}
\subsection*{Modificación de la arquitectura}
\subsection*{Comparativa de modelos de predicción}
\subsection*{Desarrollo del servicio}

% ESTE TITULO ES TEMPORAL
\section*{Puesta en marcha}
\capitulo{6}{Trabajos relacionados}

En un entorno industrial, aumentar la eficiencia de los sistemas utilizados es un proceso esencial, pues permite 
mejorar la productividad de los procesos y reducir costes. Por ello, hay una gran cantidad de artículos científicos 
que tratan temas de mantenimiento predictivo utilizando técnicas como el aprendizaje automático.

A continuación, se muestran varios artículos que tratan este problema, ya sea en entornos en los que se utilicen
AGV, o entornos puramente industriales.
\paragraph{Predictive maintenance enabled by machine learning: Use cases and challenges in the automotive industry}
En este artículo \cite{THEISSLER2021107864}, se realiza un estudio y clasificación de estudios que proponen la aplicación de aprendizaje 
automático para la realización de mantenimiento predictivo de en el campo de la industria automovilística.
Se proponen retos a resolver, así como posibles direcciones de investigación. Las conclusiones obtenidas finalmente 
son que disponer de datos públicos impulsaría las actividades de investigación, la mayoría de los artículos se basan 
en métodos supervisados que requieren datos etiquetados, la combinación de múltiples fuentes de datos puede 
mejorar la precisión, y el uso de métodos de aprendizaje profundo seguirá aumentando, pero requiere métodos 
eficientes e interpretables y la disponibilidad de grandes cantidades de datos.

\paragraph{Availability assessment for a multi-AGV system based on simulation modeling approach}
Este artículo \cite{9590979} propone el desarrollo de una simulación de Monte Carlo de un sistema multi-AGV 
utilizado para evaluar la disponibilidad del sistema. El análisis de sensibilidad para investigar las relaciones 
directas entre el mantenimiento y los parámetros operativos y el nivel de ratio de disponibilidad se 
realiza basándose en el modelo desarrollado. El análisis detallado de los logros en este ámbito permite 
identificar las lagunas en la investigación y las posibles líneas de investigación futuras para los 
procesos de diseño de almacenes autónomos.

\paragraph{Combining empirical mode decomposition and deep recurrent neural networks for predictive maintenance of lithium-ion battery}
Los autores proponen \cite{CHEN2021101405} un método híbrido de ciencia de datos basado en la descomposición empírica de modos, el análisis relacional gris 
y las redes neuronales recurrentes profundas para la predicción de la vida útil restante de las baterías de iones de litio.
Los resultados experimentales obtenidos con los datos de las baterías de iones de litio del Repositorio de 
Datos de Prognosis Ames de la NASA muestran que el modelo híbrido de ciencia de datos propuesto puede 
predecir con precisión el estado de salud y la vida útil restante de las baterías de iones de litio.

\paragraph{Novel methodology for optimising the design, operation and maintenance of a multi-AGV system}
Según los autores \cite{YAN2018130}, los problemas de fiabilidad y las estrategias de mantenimiento de los AGV no se han estudiado suficientemente,
por lo que han realizado una investigación considerando un sistema multi-AGV, compuesto por tres AGVs, con 
el fin de desarrollar una metodología científica para optimizar el diseño del layout, la operación y el 
mantenimiento de un sistema multi-AGV. Los resultados de simulación obtenidos muestran claramente que la 
ubicación de los puntos de mantenimiento y las estrategias de mantenimiento tienen una influencia 
significativa en el rendimiento de un sistema multi-AGV, donde el mantenimiento correctivo es una medida 
eficaz para mantener la fiabilidad y estabilidad del sistema a largo plazo.
\capitulo{7}{Conclusiones y líneas de trabajo futuras}

A continuación, se exponen las conclusiones obtenidas tras la finalización del proyecto, así como mejoras a realizar 
en el futuro.

\section{Conclusiones}

El objetivo general del proyecto ha sido cumplido de manera satisfactoria: se ha conseguido el desarrollo de un sistema
capaz de almacenar los datos recibidos por un AGV o por el simulador y de llevar a cabo predicciones precisas que permitan 
realizar mantenimiento predictivo.

Se ha desarrollado un sistema modular, basado en el desarrollo de microservicios independientes entre sí. Gracias a
esto, se ha conseguido también que el sistema se adapte a las limitaciones de hardware del cliente.

Gracias a la utilización de Python y Docker, se ha conseguido crear un sistema sobre el que es muy fácil de realizar 
modificaciones a futuro: Python agiliza mucho el desarrollo al ser un lenguaje interpretado, que no necesita de compilaciones 
que resten tiempo al proceso de despliegue, y Docker permite modificar solo aquellos servicios en los que haya cambios.

Durante el desarrollo del proyecto, se han utilizado metodologías y conocimientos obtenidos durante el grado. Se 
ha seguido una metodología ágil para el desarrollo, se han aprovechado los conocimientos obtenidos sobre bases de 
datos para la elección del sistema gestor de bases de datos, se han utilizado los conocimientos obtenidos sobre 
inteligencia artificial y aprendizaje automático, etc.

El desarrollo del proyecto ha servido también para ampliar mis conocimientos en dichos campos. Me ha servido 
también para adquirir nuevos conocimientos, como herramientas de Integración continua, nuevos métodos de aprendizaje 
automático, uso de contenedores de Docker para la creación de microservicios, etc.

Por el trabajo de investigación realizado durante el desarrollo del proyecto, mi capacidad para buscar información 
de calidad ha mejorado de manera sustancial, así como mi capacidad para leer y redactar documentos científicos.

Quiero destacar también la labor realizada por mis tutores, guiándome de manera efectiva durante todo el desarrollo del 
trabajo. También, por sugerencia suya, una parte de este trabajo ha sido publicado en la conferencia SOCO 2023 \cite{8364SOCO}.

\section{Líneas de trabajo futuras}


Con el fin de mejorar la modularidad del proyecto, se sugieren los siguientes cambios:
\begin{itemize}
    \item Crear un nuevo servicio que haga de intermediario para otros servicios que quieran interaccionar con la propia base de datos 
        (Figura \ref{fig:arquitectura_futura}). Para ello, se propone el desarrollado de una API Rest, mediante la cual 
        se hagan las peticiones a este servicio para hacer consultas e inserciones. De esta forma, se consigue 
        desacoplar los diferentes servicios de la base de datos, lo que resultaría en un diseño mucho más modular, y 
        que simplificaría mucho el desarrollo de nuevos servicios en el futuro.
    \item Modificar el servicio ``Forecaster'', para que pueda detectar comportamientos anómalos de los AGV de manera 
        automática utilizando la detección de anomalías de Darts \cite{darts:anomaly}, herramienta utilizada para realizar las predicciones.
    \item Modificar el servicio ``Simulator'' de manera que permita simular más de un AGV. De manera similar, se sugiere 
        modificar también el servicio ``Forecaster'' para que pueda realizar predicciones de varios vehículos.
    \item Ejecutar la optimización de los modelos de predicción comparados durante más tiempo. Por limitaciones 
        de plazos, no fue posible optimizar dichos modelos durante mucho tiempo, por lo que todavía se pueden obtener 
        mejores modelos potenciales.
\end{itemize}

\imagen{arquitectura_futura}{Diseño futuro de la arquitectura}{1}


% \bibliographystyle{ieeetr}
% \bibliography{bibliografia}

\printbibliography

\end{document}
