\apendice{Documentación técnica de programación}

\section{Introducción}

En este apartado del anexo se detallará la organización de los repositorios del proyecto, así
como un manual de programador, una guía de instalación y ejecución del mismo y las pruebas realizadas
al sistema.

\section{Estructura de directorios}

La organización de los directorios queda definida de la siguiente manera:

\dirtree{%
.1 memoria.
.2 build.
.2 img.
.2 tex.
}

El directorio memoria guarda todos los archivos relacionados con la memoria y estos anexos. En 
el subdirectorio build se encuentran los archivos pdf finales, en img se encuentran las imágenes 
utilizadas y en tex se guardan los archivos de LaTeX de cada uno de las secciones de la memoria y 
los anexos.

\dirtree{%
.1 prototipos.
.2 datos.
.2 forecasting.
.2 graphite.
.2 influxdb.
.2 mongo.
.2 prometheus.
.2 sqlite.
.2 timescaledb.
}

En el directorio prototipos se guardan todos los prototipos creados durante la realización del 
proyecto. Es código realizado de forma rápida con el fin de adecuarme al uso de cada una de las 
diferentes herramientas consideradas. No tiene mucha más importancia más allá de eso.

\dirtree{%
.1 rendimiento.
.2 database.
.2 forecast.
}

En el directorio de rendimiento se guarda todo el código de las pruebas de rendimiento realizadas 
para la comparación de las bases de datos y modelos de predicción. En el subdirectorio de 
forecast se encuentra también el código utilizado para la optimización de los hiperparámetros de 
los modelos comparados.

\dirtree{%
.1 servicios.
.2 forecasting.
.3 src.
.2 reciever.
.3 src.
.2 simulator.
.3 src.
.3 data.
}

En el directorio de servicios se encuentra todo el código y archivos de configuración de cada 
uno de los servicios desarrollados en este proyecto. El código se encuentra en el directorio
``src'' de dentro de cada uno de los subdirectorios de cada servicio. En la carpteta ``data'',
dentro del directorio del simulador, se guardan los archivos csv utilizados para leer desde 
este servicio.

\section{Manual del programador}

\section{Compilación, instalación y ejecución del proyecto}



\section{Pruebas del sistema}
