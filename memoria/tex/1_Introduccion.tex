\capitulo{1}{Introducción}

% Descripción del contenido del trabajo y del estrucutra de la memoria y del resto de materiales entregados.

Los AGV, Autonomous Guided Vehicles por sus siglas en inglés, son complejos sistemas robóticos, 
capaces de moverse en un entorno concreto, cuyo uso es transportar cargas pesadas en fábricas o 
almacenes, y que están diseñados para mejorar la eficiencia y la productividad en la logística 
y el transporte de materiales. Debido a sus ventajas en seguridad, flexibilidad y velocidad,
esta tecnología se está convirtiendo cada vez más importante \cite{espinosa2020transporte}.

Aunque estos sistemas pueden mejorar la productividad, desajustes en su configuración u otros
errores operacionales pueden producir una reducción de su rendimiento, y, en casos extremos,
causar una detención de la línea de producción. Por este motivo, es necesario extraer información
de los sistemas en marcha para analizar el rendimiento de las máquinas y las aplicaciones logísticas.
Esta información puede usarse para predecir comportamientos futuros del sistema, realizar mantenimiento
predictivo y proveer retroalimentación con el fin de diseñar mejoras continuas de las máquinas. Estas
predicciones pueden ser conseguidas con el uso de algoritmos de análisis de series temporales, que
permitan anticipar futuras condiciones del sistema. \cite{BARUQUE201949}

Algunos de estos datos obtenidos del sistema tienen una baja frecuencia de actualización, como puede
ser la temperatura y voltaje de la batería, pero otros cambian cada pocos milisegundos, como la
corriente eléctrica, la velocidad, la posición del vehículo, errores y estado, etc. Toda esta
información proveída por el AGV debe estar relacionada con el tiempo en el que fue generada, por
lo que puede ser agrupada en series temporales \cite{DBLP:journals/corr/abs-2104-00164}. Cualquier
tipo de base de datos puede usarse para almacenar esta información generada por los AGV, sin embargo,
ya que se trata de series temporales, es preferible utilizar bases de datos para series temporales
para optimizar el rendimiento del sistema.

Estos datos pueden ser posteriormente analizados y utilizados para entrenar un modelo basado en técnicas 
estadísticas o en redes neuronales, que sirva para predecir los indicadores de rendimiento de estos vehículos.
Con estas predicciones, se pueden detectar errores con mucha mayor antelación, lo que puede permitir tomar 
mediadas que no sería factible tomar de otra manera. De esta forma, se pueden conseguir reducir el número 
de errores críticos que supongan una parada del sistema, reduciendo considerablemente los costes y aumentando 
la productividad.

Este proyecto está enfocado a utilizarse en un entorno industrial, por lo que su principal objetivo es claro:
reducir costes a partir de una mejora de la eficiencia del sistema. Por ello, escoger la base de datos más 
óptima para esta tarea, así como el mejor modelo para realizar las predicciones son tareas esenciales. Es por 
esto que este trabajo está muy centrado precisamente en dichas tareas de investigación.

\subsection{Estructura de la memoria}

Esta memoria seguirá la siguiente estructura:
\begin{itemize}
    \item \textbf{Introducción:} contiene una breve descripción del trabajo, así como una guía de contenidos del mismo.
    \item \textbf{Objetivos del proyecto:} detalla los objetivos principales del proyecto.
    \item \textbf{Conceptos teóricos:} explicación de varios conceptos necesarios para una mejor comprensión del
        proyecto.
    \item \textbf{Técnicas y herramientas:} metodologías y técnicas utilizadas durante todo el desarrollo del 
        trabajo.
    \item \textbf{Aspectos relevantes del desarrollo:} descripción del proceso de desarrollo, con los aspectos
        más relevantes del mismo.
    \item \textbf{Trabajos relacionados:} exposición y comparativa con otros trabajos relacionados.
    \item \textbf{Conclusiones y líneas de trabajo futuras:} conclusiones obtenidas durante el desarrollo, así
        como ideas a futuro para la mejora del mismo.
\end{itemize}

Junto a la memoria, se incluye a demás unos anexos que incluyen:
\begin{itemize}
    \item \textbf{Plan de proyecto:} contiene la planificación temporal y un estudio de viabilidad legal y
        económica.
    \item \textbf{Requisitos:} describe los requisitos funcionales y no funcionales del sistema, así como una
        comparativa de gestores de bases de datos y modelos de predicción, basándose en dichos requisitos.
    \item \textbf{Diseño:} se define el diseño de los datos, procedimental y arquitectónico del sistema.
    \item \textbf{Manual del programador:} recoge los aspectos más importantes para futuros programadores del sistema.
    \item \textbf{Manual de usuario:} detalla las funcionalidades del proyecto para futuros usuarios del
        mismo.
\end{itemize}

\subsection{Materiales adjuntos}

Junto con esta memoria y anexos, se incluye:
\begin{itemize}
    \item Prototipos. Recoge el conjunto de prototipos realizados durante el desarrollo.
    \item Análisis de rendimiento. Contiene los programas utilizados para hacer las pruebas de rendimiento 
        de las bases de datos y de los modelos de predicción.
    \item Código del sistema desarrollado.
    \item Conjunto de datos de prueba. Se incluye, incluido en los directorios de los servicios, un conjunto de datos de un AGV 
        real con el que realizar pruebas.
\end{itemize}