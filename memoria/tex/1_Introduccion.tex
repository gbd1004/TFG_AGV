\capitulo{1}{Introducción}

% Descripción del contenido del trabajo y del estrucutra de la memoria y del resto de materiales entregados.

Los AGV, Autonomous Guided Vehicles por sus siglas en inglés, son complejos sistemas robóticos, 
capaces de moverse en un entorno concreto, cuyo uso es transportar cargas pesadas en fábricas o 
almacenes, y que están diseñados para mejorar la eficiencia y la productividad en la logística 
y el transporte de materiales. Debido a sus ventajas en seguridad, flexibilidad y velocidad,
esta tecnología se está convirtiendo cada vez más importante \cite{espinosa2020transporte}.

Aunque estos sistemas pueden mejorar la productividad, desajustes en su configuración u otros
errores operacionales pueden producir una reducción de su rendimiento, y, en casos extremos,
causar una detención de la línea de producción. Por este motivo, es necesario extraer información
de los sistemas en marcha para analizar el rendimiento de las máquinas y las aplicaciones logísticas.
Esta información puede usarse para predecir comportamientos futuros del sistema, realizar mantenimiento
predictivo y proveer retroalimentación con el fin de diseñar mejoras continuas de las máquinas. Estas
predicciones pueden ser conseguidas con el uso de algoritmos de análisis de series temporales, que
permitan anticipar futuras condiciones del sistema. \cite{BARUQUE201949}

Algunos de estos datos obtenidos del sistema tienen una baja frecuencia de actualización, como puede
ser la temperatura y voltaje de la batería, pero otros cambian cada pocos milisegundos, como la
corriente eléctrica, la velocidad, la posición del vehículo, errores y estado, etc. Toda esta
información proveída por el AGV debe estar relacionada con el tiempo en el que fue generada, por
lo que puede ser agrupada en series temporales \cite{DBLP:journals/corr/abs-2104-00164}. Cualquier
tipo de base de datos puede usarse para almacenar esta información generada por los AGV, sin embargo,
ya que se trata de series temporales, es preferible utilizar bases de datos para series temporales
para optimizar el rendimiento del sistema.

