\apendice{Plan de Proyecto Software}

\section{Introducción}

En el ámbito del desarrollo de proyectos de software, es esencial contar con una planificación
sólida y realista, así como evaluar la viabilidad económica y legal del proyecto. Estos aspectos 
son clave para garantizar el éxito a largo plazo, la rentabilidad y la conformidad con las 
regulaciones aplicables.

La fase de planificación se divide por tanto en los siguientes pasos:
\begin{itemize}
    \item Planificación temporal: se organizará el trabajo en los sprints correspondientes.
    \item Estudio de viabilidad: se divide a su vez en otros dos pasos:
    \begin{itemize}
        \item Viabilidad económica: se explorarán los gastos del desarrollo del proyecto 
            y los posibles beneficios generados por el mismo.
        \item Viabilidad legal: se discutirá la licencia del proyecto, la cual depende de las
            licencias de las dependencias utilizadas.
    \end{itemize}
\end{itemize}

Para el paso de la planificación temporal, se analizará el tiempo estimado para completar cada 
uno de las partes del proyecto, y se distribuirán de manera uniforme en todos los sprints según 
las convenciones marcadas por el método SCRUM.

Para la segunda parte de la planificación, se tendrán en cuenta los gastos de hardware, software, 
personal y otros (como el precio del alquiler y el internet). Se tendrán también en cuenta el modelo de 
negocio planteado para generar beneficios.

Para ello, será necesario comprobar que las licencias de las dependencias utilizadas en el proyecto 
permiten utilizar el modelo de negocio propuesto, por lo que deberá realizarse un análisis de la 
viabilidad legal del proyecto.

\section{Planificación temporal}

% TODO

\section{Estudio de viabilidad}

\subsection{Viabilidad económica}

En este apartado, se analizan los costes y beneficios que podría haber incurrido con el desarrollo 
de este proyecto en el caso de que se hubiese desarrollado en un entorno profesional.

\subsubsection{Costes}

\paragraph{Costes materiales}
En este apartado se analizan todos los costes materiales relacionados con el proyecto, principalmente 
hardware. Se estima una vida útil del hardware de 4 años, habiéndose utilizado 6 meses para 
el desarrollo de este proyecto.

%TODO MIRAR A VER SI AÑADIR SERVIDOR
\tablaSmallSinColores{Costes de hardware}{l l l}{coste_hw}{
    Elemento & Coste & Coste amortizado \\
}{
    Ordenador & 1500 € & 375 € \\
    Periféricos & 100 € & 25 € \\ 
    \midrule
    Total & 1500 € & 400 € \\
}

\paragraph{Costes de software}
Debido a que el proyecto ha sido desarrollado utilizando únicamente software de código abierto,
los costes de software han sido de cero. Convendría, sin embargo, en el supuesto caso de aplicar
este proyecto en un entorno industrial real, estudiar si merece la pena utilizar la versión empresarial 
y de pago de InfluxDB, la base de datos utilizada.

\paragraph{Costes de personal}

El desarrollo del proyecto ha sido llevado a cabo por un único programador. Se supone que ha sido 
a jornada completa y que se trata de un desarrollador junior.

\tablaSmallSinColores{Costes de personal}{l l}{coste_per}{
    Elemento & Coste \\
}{
    Salario bruto anual & 18.000 € \\
    IRPF anual (13,25\%) & 2.385 € \\
    Seguridad Social & 1.269 € \\
    Salario neto anual & 14.346 € \\
    \midrule
    Total 6 meses & 7.173 € \\
}

\paragraph{Otros costes}

En este apartado se estudian otros costes como el alquiler, internet, luz, etc.

\tablaSmallSinColores{Otros costes}{l l}{otros_costes}{
    Elemento & Coste \\
}{
    Alquiler (mensual) & 300 € \\
    Internet (mensual) & 50 € \\
    Luz (mensual) & 50 € \\
    Calefacción (mensual) & 40 € \\ 
    Agua (mensual) & 17 € \\
    \midrule
    Total 6 meses & 2.742 € \\
}

\paragraph{Coste total}

La suma de todos los costes queda reflejada en la siguiente tabla (Tabla \ref{tabla:suma_costes})

\tablaSmallSinColores{Otros costes}{l l}{suma_costes}{
    Elemento & Coste \\
}{
    Hardware & 400 € \\
    Software & 0 € \\
    Personal & 7.173 € \\
    Otros & 2.742 € \\
    \midrule
    Total & 10.315 € \\
}

\subsubsection{Beneficios}

Ya que el sistema desarrollado se distribuye principalmente a empresas, el precio de adquisición del
mismo será de 5.000 €. Se ofrece también a demás un servicio de soporte continuo con un precio de 
500 € al mes.

% TODO Revisar que encaje con las licencias

\subsection{Viabilidad legal}

En este apartado se analizan las licencias del software utilizado, y se detallará la licencia 
escogida para el software del proyecto, así como de la documentación, imágenes y vídeos utilizados.

\subsubsection{Sofware}

Se muestran a continuación las licencias del software utilizado.

\tablaSmallFija{Licencias de las dependencias}{l l l}{licencias}{
Dependencia & Versión & Licencia \\
}{
InfluxDB & 2.6.1 & MIT \\
Docker & 24.0.2 & Docker Subscription Service Agreement \\
Docker compose & 2.16.0 & Apache License 2.0 \\
Nvidia-docker & 2.13.0 & Apache License 2.0 \\
Darts & 0.24.0 & Apache License 2.0 \\
Python UDP & 0.5.10 & MIT \\
Python & 3.10 & Python Software Foundation \\ 
CUDA & 11.8 & NVIDIA Deep Learning Container license\\
}
