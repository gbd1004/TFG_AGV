\apendice{Plan de Proyecto Software}

\section{Introducción}

En el ámbito del desarrollo de proyectos de software, es esencial contar con una planificación
sólida y realista, así como evaluar la viabilidad económica y legal del proyecto. Estos aspectos 
son clave para garantizar el éxito a largo plazo, la rentabilidad y la conformidad con las 
regulaciones aplicables.

La fase de planificación se divide por tanto en los siguientes pasos:
\begin{itemize}
    \item Planificación temporal: se organizará el trabajo en los sprints correspondientes.
    \item Estudio de viabilidad: se divide a su vez en otros dos pasos:
    \begin{itemize}
        \item Viabilidad económica: se explorarán los gastos del desarrollo del proyecto 
            y los posibles beneficios generados por el mismo.
        \item Viabilidad legal: se discutirá la licencia del proyecto, la cual depende de las
            licencias de las dependencias utilizadas.
    \end{itemize}
\end{itemize}

Para el paso de la planificación temporal, se analizará el tiempo estimado para completar cada 
uno de las partes del proyecto, y se distribuirán de manera uniforme en todos los sprints según 
las convenciones marcadas por el método SCRUM.

Para la segunda parte de la planificación, se tendrán en cuenta los gastos de hardware, software, 
personal y otros (como el precio del alquiler y el internet). Se tendrán también en cuenta el modelo de 
negocio planteado para generar beneficios.

Para ello, será necesario comprobar que las licencias de las dependencias utilizadas en el proyecto 
permiten utilizar el modelo de negocio propuesto, por lo que deberá realizarse un análisis de la 
viabilidad legal del proyecto.

\section{Planificación temporal}

Debido a la flexibilidad que ofrece SCRUM \cite{schwaber2001agile}, una metodología ágil, se planteó desde el inicio seguir dicha 
metodología. Precisamente gracias a dicha flexibilidad, no se ha seguido al completo. Por ejemplo, ya que 
el proyecto solo ha sido desarrollado por mí, no ha sido necesario la realización de una reunión diaria.

Como marca esta metodología, el trabajo se divide en ``sprints'', de unas dos semanas de duración, en la que 
se planifica lo que se va a realizar en una reunión al comienzo de dicho sprint. Al final de cada sprint, se
ha realizado otra reunión (normalmente solapada con la del inicio del siguiente) para revisar el trabajo realizado.

Para el seguimiento de los sprints, inicialmente se empezó a usar la herramienta ZenHub. Sin embargo, por un problema
con las licencias, se dejó de usar en mitad del proyecto. Para el control de los sprints, se empezó a usar en su 
lugar el apartado de ``Issues'' de GitHub. Para ello, se crea un issue con la tarea a realizar y se asigna una estimación 
temporal. Dicha asignación temporal puede verse reflejada en la Tabla \ref{tabla:asignacion_temp}

\tablaSmallFija{Asignación temporal}{l l}{asignacion_temp}{
    Story points & Estimación \\
}{
    1 & 30 minutos \\
    2 & 1 hora \\
    3 & 2 horas \\
    5 & 3 horas \\
    8 & 6 horas \\
    13 & 12 horas \\
    21 & 1 día \\
    34 & 3 día \\
    55 & 1 semana \\
}

Se muestra a continuación la planificación de cada Sprint.

\subsection{Sprint 1 (2 de marzo de 2023 - 14 de marzo de 2023)}

En este Sprint se comenzó el proyecto, quedando definidos los objetivos del mismo. Este Sprint se dedicó principalmente 
a tareas de investigación: se investigaron diferentes sistemas gestores de bases de datos, herramientas de visualización 
de datos, ElasticSearch como recomendación de los tutores y Docker; y a tareas de inicialización del repositorio: se 
importó la plantilla de LaTeX de la memoria y anexos y se desarrollaron las versiones iniciales del servicio ``Simulator'' y 
``Receiver''.

\imagen{sprint1}{Burndown del Sprint 1}

\subsection{Sprint 2 (15 de marzo de 2023 - 31 de marzo de 2023)}

Este Sprint se dedicó a la mejora de los servicios realizados en el Sprint anterior, así como la redacción de aspectos 
generales de la memoria. Se integraron también dichos servicios con la herramienta docker compose para ejecutarlos todos 
de manera coordinada. Por último, en este Sprint se comenzó a escribir lo publicado en el SOCO.

\imagen{sprint2}{Burndown del Sprint 2}

\subsection{Sprint 3 (31 de marzo de 2023 - 1 de mayo de 2023)}

La duración de este Sprint tuvo una mayor duración, y una menor carga de trabajo debido a dos motivos. Por un lado
coincidió las vacaciones de Semana Santa. Por otro lado, en este Sprint me centré principalmente en escribir el informe 
del SOCO, por lo que no hubo muchos avances directos en este proyecto.

\imagen{sprint3}{Burndown del Sprint 3}

\subsection{Sprint 4 (1 de mayo de 2023 - 15 de mayo de 2023)}

Al igual que en el anterior Sprint, este se centró también en la redacción del trabajo presentado en el SOCO, por lo 
que este Sprint se centró en dicha tarea y en realizar avances en la memoria.

\imagen{sprint4}{Burndown del Sprint 4}

\subsection{Sprint 5 (15 de mayo de 2023 - 31 de mayo de 2023)}

Una vez terminado el informe del SOCO, el trabajo de este Sprint se centró en tres apartados: integrar dicho trabajo 
en la memoria y anexos del proyecto, investigar herramientas de aprendizaje automático e investigar y optimizar 
los modelos escogidos en dicha investigación.

\imagen{sprint5}{Burndown del Sprint 5}

\subsection{Sprint 6 (31 de mayo de 2023 - 15 de junio de 2023)}

En este Sprint, se desarrolló el servicio ``Forecaster'' y se mejoró el simulador para que pudiese leer datos 
desde un archivo CSV. Se continuó integrando el trabajo del SOCO, así como mejorar el README del proyecto y realizar 
avances en la memoria y anexos.

\imagen{sprint6}{Burndown del Sprint 6}

\subsection{Sprint 7 (15 de junio de 2023 - 4 de julio de 2023)}

Como último Sprint, el trabajo realizado en este Sprint se centró en la mejora del código y de la funcionalidad 
de los servicios, así como terminar la memoria y anexos. Se integró Code Climate en el repositorio para monitorizar 
la mantenibilidad del código y Pylint para comprobar que se siguen las convenciones estilísticas. Por último, se 
añadió la opción de usar GPU o CPU para entrenar el modelo de predicción.

\imagen{sprint7}{Burndown del Sprint 7}

\section{Estudio de viabilidad}

\subsection{Viabilidad económica}

En este apartado, se analizan los costes y beneficios que podría haber incurrido con el desarrollo 
de este proyecto en el caso de que se hubiese desarrollado en un entorno profesional.

\subsubsection{Costes}

\paragraph{Costes materiales}
En este apartado se analizan todos los costes materiales relacionados con el proyecto, principalmente 
hardware. Se estima una vida útil del hardware de 4 años, habiéndose utilizado 6 meses para 
el desarrollo de este proyecto.

%TODO MIRAR A VER SI AÑADIR SERVIDOR
\tablaSmallSinColores{Costes de hardware}{l l l}{coste_hw}{
    Elemento & Coste & Coste amortizado \\
}{
    Ordenador & 1500 € & 375 € \\
    Periféricos & 100 € & 25 € \\ 
    \midrule
    Total & 1500 € & 400 € \\
}

\paragraph{Costes de software}
Debido a que el proyecto ha sido desarrollado utilizando únicamente software de código abierto,
los costes de software han sido de cero. Convendría, sin embargo, en el supuesto caso de aplicar
este proyecto en un entorno industrial real, estudiar si merece la pena utilizar la versión empresarial 
y de pago de InfluxDB, la base de datos utilizada.

\paragraph{Costes de personal}

El desarrollo del proyecto ha sido llevado a cabo por un único programador. Se supone que ha sido 
a jornada completa y que se trata de un desarrollador junior.

\tablaSmallSinColores{Costes de personal}{l l}{coste_per}{
    Elemento & Coste \\
}{
    Salario bruto anual & 18.000 € \\
    Seguridad Social & 6.000 € \\
    Total anual & 24.000 € \\
    \midrule
    Total 6 meses & 12.000 € \\
}

\paragraph{Otros costes}

En este apartado se estudian otros costes como el alquiler, internet, luz, etc.

\tablaSmallSinColores{Otros costes}{l l}{otros_costes}{
    Elemento & Coste \\
}{
    Alquiler (mensual) & 300 € \\
    Internet (mensual) & 50 € \\
    Luz (mensual) & 50 € \\
    Calefacción (mensual) & 40 € \\ 
    Agua (mensual) & 17 € \\
    \midrule
    Total 6 meses & 2.742 € \\
}

\paragraph{Coste total}

La suma de todos los costes queda reflejada en la siguiente tabla (Tabla \ref{tabla:suma_costes})

\tablaSmallSinColores{Otros costes}{l l}{suma_costes}{
    Elemento & Coste \\
}{
    Hardware & 400 € \\
    Software & 0 € \\
    Personal & 7.173 € \\
    Otros & 2.742 € \\
    \midrule
    Total & 10.315 € \\
}

\subsubsection{Beneficios}

Ya que el sistema desarrollado se distribuye principalmente a empresas, el precio de adquisición del
mismo será de 5.000 €. Se ofrece también a demás un servicio de soporte continuo con un precio de 
500 € al mes.

\subsection{Viabilidad legal}

En este apartado se analizan las licencias del software utilizado, y se detallará la licencia 
escogida para el software del proyecto, así como de la documentación, imágenes y vídeos utilizados.

\subsubsection{Sofware}

Una licencia es ``un contrato entre el desarrollador de un software sometido a la propiedad
intelectual y a derechos de autor y el usuario, en el cual se definen con precisión los 
derechos y deberes de ambas partes.'' \cite{labrador2012tipos}
Se muestran en la Tabla \ref{tabla:licencias} las licencias del software utilizado.

\tablaSmall{Licencias de las dependencias}{l l l}{licencias}{
Dependencia & Versión & Licencia \\
}{
InfluxDB & 2.6.1 & MIT \\
Docker & 24.0.2 & Docker Subscription Service Agreement \\
Docker compose & 2.16.0 & Apache License 2.0 \\
Nvidia-docker & 2.13.0 & Apache License 2.0 \\
Darts & 0.24.0 & Apache License 2.0 \\
Python UDP & 0.5.10 & MIT \\
Python & 3.10 & Python Software Foundation \\ 
CUDA & 11.8 & NVIDIA Deep Learning Container license\\
}

Inicialmente se planteó utilizar la licencia GPLv3 \cite{gplv3}, sin embargo, esta licencia 
obliga a que todo el código del programa y las herramientas utilizadas cumplan esta licencia: 
``You must license the entire work, as a whole, under this License to anyone who comes into 
possession of a copy. This License will therefore apply, along with any applicable section 
7 additional terms, to the whole of the work, and all its parts, regardless of how they are 
packaged.'' Esto sin embargo no es compatible con la licencia NVIDIA Deep Learning Container license
\cite{nvidia_dl_container_license}: ``You may not use the CONTAINER in any manner that would cause it
to become subject to an open source software license.'' Por ello, finalmente se decidió 
usar la licencia MIT \cite{mit_license} para todo el código del proyecto.

\subsubsection{Documentación}

Para licenciar la memoria y estos anexos se ha decidido utilizar una licencia ``Creative Commons''.
Para elegir la más adecuada, se ha utilizado la herramienta ``License Chooser'' \cite{creativecommons_chooser}.
Finalmente, se ha elegido la licencia ``Attribution-ShareAlike 4.0 International (CC BY-SA 4.0)'' \cite{cc_by_sa_license}, que permite 
tanto el uso comercial, como compartir dicha documentación, siempre y cuando se compartan de la misma manera que 
en este trabajo.

\subsubsection{Imágenes y vídeos}

Para licenciar las imágenes y vídeos se ha utilizado la licencia ``Attribution-ShareAlike 4.0 International (CC BY-SA 4.0)'', 
la misma que la de la documentación.