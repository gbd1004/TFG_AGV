\apendice{Plan de Proyecto Software}

\section{Introducción}

En el ámbito del desarrollo de proyectos de software, es esencial contar con una planificación
sólida y realista, así como evaluar la viabilidad económica y legal del proyecto. Estos aspectos 
son clave para garantizar el éxito a largo plazo, la rentabilidad y la conformidad con las 
regulaciones aplicables.

El presente proyecto de software se enfoca en el análisis detallado de la planificación temporal,
la viabilidad económica y la viabilidad legal, con el objetivo de proporcionar una base sólida
para su implementación exitosa. Se realizará un exhaustivo examen de cada uno de estos aspectos,
considerando tanto los recursos disponibles como las restricciones y requisitos específicos del
proyecto.

En primer lugar, se llevará a cabo un análisis de la planificación temporal del proyecto. Esto
implica establecer un cronograma realista y eficiente, teniendo en cuenta las tareas necesarias,
los hitos clave y las dependencias entre ellas. Se identificarán los recursos requeridos y se
asignarán de manera adecuada para asegurar una ejecución eficiente y oportuna del proyecto. 
Además, se considerarán posibles riesgos y se desarrollarán estrategias de mitigación para 
evitar retrasos y garantizar la finalización exitosa del software en el tiempo establecido.

En segundo lugar, se realizará un estudio exhaustivo de la viabilidad económica del proyecto 
de software. Esto implica evaluar los costos asociados con el desarrollo, implementación y 
mantenimiento del software, así como proyectar los beneficios y retornos de inversión esperados. 
Se analizarán los posibles ingresos generados por el software, los gastos operativos, el 
costo de adquisición de herramientas y tecnologías necesarias, entre otros factores 
relevantes. Este análisis permitirá tomar decisiones informadas sobre la asignación de 
recursos financieros y evaluar la rentabilidad y sostenibilidad del proyecto.

Por último, pero no menos importante, se llevará a cabo un análisis exhaustivo de la viabilidad 
legal del proyecto de software. Se examinarán las leyes y regulaciones pertinentes, como la 
protección de datos, los derechos de propiedad intelectual y cualquier otro requisito legal 
aplicable. Se asegurará que el software cumpla con todas las normativas y se evitarán posibles 
problemas legales en el futuro. Además, se considerarán las implicaciones éticas y de privacidad 
para garantizar el cumplimiento de estándares éticos y legales relevantes.

En resumen, este proyecto de software se enfoca en el análisis riguroso de la planificación 
temporal, la viabilidad económica y la viabilidad legal. Al considerar estos aspectos de 
manera integral, se establecerán las bases necesarias para el desarrollo exitoso del software, 
asegurando su viabilidad financiera, su cumplimiento legal y su capacidad para cumplir con los 
objetivos establecidos.

\section{Planificación temporal}

% TODO

\section{Estudio de viabilidad}

\subsection{Viabilidad económica}

En este apartado, se analizan los costes y beneficios que podría haber incurrido con el desarrollo 
de este proyecto en el caso de que se hubiese desarrollado en un entorno profesional.

\subsubsection*{Costes}

\paragraph*{Costes materiales}
En este apartado se analizan todos los costes materiales relacionados con el proyecto, principalmente 
hardware. Se estima una vida útil del hardware de 4 años, habiéndose utilizado 6 meses para 
el desarrollo de este proyecto.

%TODO MIRAI A VER SI AÑADIR SERVIDOR
\tablaSmallSinColores{Costes de hardware}{l l l}{coste_hw}{
    Elemento & Coste & Coste amortizado \\
}{
    Ordenador & 1500€ & 375€ \\
    \midrule
    Total & 1500€ & 375€ \\
}

\paragraph*{Costes software}
Debido a que el proyecto ha sido desarrollado utilizando únicamente software de código abierto,
los costes de software han sido de cero. Convendría, sin embargo, en el supuesto caso de aplicar
este proyecto en un entorno industrial real, estudiar si merece la pena utilizar la versión empresarial 
y de pago de la base de datos utilizada, InfluxDB.

\paragraph*{Costes personales}

El desarrollo del proyecto ha sido llevado a cabo por un único programador. Se supone que ha sido 
a jornada completa y que se trata de un desarrollador junior.

\tablaSmallSinColores{Costes personales}{l l}{coste_per}{
    Elemento & Coste \\
}{
    Salario & 1200€ \\
    IRPF & \\
    Seguridad Social & \\
    Salario bruto & \\
    \midrule
    Total 6 meses & \\
}

\paragraph*{Otros costes}

% TODO ALQUILER, INTERNET, LUZ, AGUA, ETC

\paragraph*{Coste total}

% TODO TOTAL

\subsubsection*{Beneficios}

\subsection{Viabilidad legal}


