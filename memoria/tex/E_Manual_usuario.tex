\apendice{Documentación de usuario}

\section{Introducción}

Este manual detalla los requisitos desde el punto de vista del usuario, así como la guía de instalación y 
un manual del funcionamiento del proyecto.

\section{Requisitos de usuarios}

\subsection{Requisitos de hardware}
Como ya ha sido mencionado con anterioridad, es muy recomendable utilizar la GPU para acelerar la predicción de 
los datos. Para ello, se necesita una GPU con soporte para CUDA 11.8 o superior. Dicha compatibilidad puede
comprobarse en \cite{cuda-compatibility}.

El otro elemento que necesita de un buen rendimiento es la base de datos. Un hardware insuficiente puede suponer 
retrasos a la hora de insertar nuevos datos e incluso cuelgues del sistema, por lo que los requisitos mínimos de 
hardware irán dictados en gran medida por dicho servicio. Según la documentación de InfluxDB \cite{influx:requirements}, para nuestras necesidades 
estimadas se necesita de una CPU de 2 a 4 núcleos y de 2 a 5 GB de memoria RAM. Un disco SSD también es muy recomendado.

\subsection{Requisitos de software}
Para la ejecución del proyecto las únicas herramientas necesarias son Docker \cite{docker-pag} y docker-compose \cite{compose}.
En cuanto al sistema operativo, el proyecto puede ejecutarse en cualquier sistema operativo que soporte estas aplicaciones.

En el caso de querer usar la versión del servicio de predicción que utiliza la GPU, es necesario también instalar 
nvidia-docker \cite{nvidia-docker}. Esta herramienta solo funciona en Linux, por lo que este es el único sistema 
operativo soportado para este caso.

\section{Instalación}

El primer paso para instalar el proyecto es clonar el repositorio con el siguiente comando:
\begin{lstlisting}[language=bash]
$ git clone https://github.com/gbd1004/TFG_AGV.git
\end{lstlisting}
Después de clonar el repositorio se navega a la carpeta ``services'' y se ejecuta el siguiente 
comando para construir las imágenes de los contenedores de cada servicio:
\begin{lstlisting}[language=bash]
$ docker compose build
\end{lstlisting}

Una vez ejecutados estos comandos tendremos el proyecto listo para ejecutarse.

\section{Manual del usuario}

\subsection{Ejecución del proyecto}

Para ejecutar el proyecto usando la GPU se utiliza el siguiente comando desde la carpeta ``services'':
\begin{lstlisting}[language=bash]
$ docker compose --profile gpu up --build
\end{lstlisting}
En caso de que no quiera, o no pueda, usarse la GPU, y por ende se utilice la CPU para el servicio de 
predicciones, se utiliza el siguiente comando:
\begin{lstlisting}[language=bash]
$ docker compose --profile cpu up --build
\end{lstlisting}
Por último, mencionar que si no quiere ejecutarse el servicio de predicción basta con ejecutar el mismo comando 
sin especificar ningún perfil.
\begin{lstlisting}[language=bash]
$ docker compose up --build
\end{lstlisting}

La opción ``--build'' construye las imágenes de los contenedores en caso de que algo haya cambiado o 
no se haya ejecutado el paso anterior.

Opcionalmente se puede añadir también el argumento ``-d'' en caso de que se quiera ejecutar desacoplado 
del terminal actual.

Para detener la ejecución del proyecto se ejecuta el siguiente comando:
\begin{lstlisting}[language=bash]
$ docker compose down
\end{lstlisting}
Se puede añadir también el argumento ``-v'' en caso de que queramos eliminar los volúmenes montados 
en los contenedores.

\subsection{Archivos de configuración}

Cada servicio cuenta con sus archivos de configuración. A continuación se detallan los elementos 
de cada uno de ellos.

\subsubsection{Database}
El archivo de configuración de la base de datos se encuentra en el directorio ``./services/database'' 
y se llama ``influxdb\_credentials.env''. Este archivo de variables de entorno se carga en el servicio 
de la base de datos y en todos los servicios que tengan conexión con dicha base de datos.

El contenido es el siguiente:
\begin{itemize}
    \item DOCKER\_INFLUXDB\_INIT\_USERNAME: Nombre de usuario utilizado para iniciar sesión en la aplicación web.
    \item DOCKER\_INFLUXDB\_INIT\_PASSWORD: Contraseña del usuario del sistema.
    \item DOCKER\_INFLUXDB\_INIT\_ADMIN\_TOKEN: Token de seguridad del usuario administrador, y que tiene permisos 
        de escritura en la base de datos.
    \item DOCKER\_INFLUXDB\_INIT\_ORG: Nombre de la organización en la que se crean los ``buckets'' y usuarios de
        la base de datos.
    \item DOCKER\_INFLUXDB\_INIT\_BUCKET: Nombre del ``bucket'' en el que se insertan los datos del AGV o del simulador.
    \item DOCKER\_INFLUXDB\_INIT\_MODE: Modo en el que se inicia la base de datos.
\end{itemize}
De todos estos campos, el único que no es configurable es el último, pues sin este valor la base de datos 
no se cargará correctamente. Tampoco se recomienda cambiar el campo DOCKER\_INFLUXDB\_INIT\_BUCKET, aunque 
el sistema no debería tener problemas en el caso de que se especifique otro nombre para el bucker de datos del AGV.

\subsubsection{Forecasting}

El archivo de configuración del servicio de predicción se encuentra en la carpeta ``./services/forecasting'', y 
se llama ``config.json''. En este archivo aparecen los siguientes campos:
\begin{itemize}
    \item load\_model: guarda un valor booleano. En caso de ser ``true'' el servicio carga el modelo a entrenar.
        Por contra, si el valor es ``false'' el modelo se entrenará en la ejecución del servicio.
    \item model\_file: nombre del modelo a cargar. Es también el nombre que tendrá el modelo guardado en el caso 
        de que se entrene el modelo.
    \item wait\_time\_before\_train: tiempo a esperar antes de entrenar para dar tiempo a que haya datos que 
        utilizar en la base de datos. Cuanto más alto sea este valor, más preciso será el modelo, pero también
        más tiempo tardará en entrenarse.
    \item wait\_time\_before\_load: tiempo a esperar antes de cargar el modelo para poder ajustar el escalador de los
        datos. Los datos necesitan escalarse, ya que el modelo de predicciones espera valores entre 0 y 1.
\end{itemize}

\subsubsection{Reciever}

El archivo de configuración para este servicio se encuentra en el directorio ``./services/reciever'' y 
contiene solo un único campo:
\begin{itemize}
    \item max\_retries: reintentos máximos que realizara este servicio en caso de que la conexión con 
        la base de datos falle al iniciar este proceso.
\end{itemize}

\subsubsection{Simulator}

El archivo de configuración del servicio de simulación, llamado también ``config.json'', se encuentra en la 
ruta ``./services/simulator''. Este archivo especifica los siguientes parámetros de configuración:
\begin{itemize}
  \item simulate: guarda un valor booleano que especifica si la simulación está activada o desactivada.
  \item from\_csv: guarda un valor booleano, que en caso de ser ``true'' carga los datos de la simulación desde
    un csv. En caso de ser ``false'' los datos generados serán completamente aleatorios.
  \item csv\_file: nombre del archivo csv a cargar en caso de que se necesite.
  \item loop: valor booleano que especifica si, en caso de simular desde un CSV, al acabar de recorrer dicho fichero 
    vuelve a ejecutarse el simulador o no.
\end{itemize}

\subsection{Entorno de usuario}

InfluxDB abre por defecto en el puerto 8086 una aplicación web, Chronograf, con la que podemos observar los datos 
almacenados. Para entrar a dicha aplicación, basta con introducir la dirección IP del ordenador en la que esté 
alojada la base de datos, seguido de dos puntos y el puerto. En el caso de que se aloje de forma local, se podrá 
acceder a Chronograf con la url ``localhost:8086''. Una vez accedido, se nos mostrará una página de inicio de sesión 
(Figrua \ref{fig:login}) en la que introduciremos lo establecido en el archivo de variables de entorno ``influxdb\_credentials.env''.

\imagen{login}{Página de inicio de sesión}

Una vez iniciada la sesión, se nos mostrará la siguiente ventana de bienvenida. En esta ventana, podemos ver en la 
parte izquierda los diferentes menús y acciones que existen.

\imagen{getting_started}{Imagen de bienvenida}

La primera de estas opciones muestra la información de la organización actual (Figura \ref{fig:about}), así como de los miembros que la forman (Figura \ref{fig:members}).
Se puede también cambiar de organización y crear nuevas organizaciones.

\imagen{about}{Información general de la organización}
\imagen{members}{Información de los miembros de la organización}

El siguiente apartado, representado por una flecha hacia arriba, es el apartado ``Load Data''. Desde aquí podremos realizar 
varias operaciones, como insertar datos, ver los ``buckets'' existentes, configurar Telegraf \cite{telegraf}, etc. En nuestro 
caso, el único apartado interesante es el de los ``buckets'' (Figura \ref{fig:ld_buckets}).

\imagen{ld_buckets}{Buckets de la base de datos}

Al seleccionar uno de estos ``buckets'', se nos redirecciona al apartado ``Data Explorer'', representado en el menú 
lateral por un símbolo de un gráfico. Desde este apartado podremos realizar consultas a la base de datos y observar los 
datos recibidos de varias formas (en este caso una gráfica). Las consultas se pueden hacer tanto de un modo gráfico (Figura \ref{fig:data_explorer}),
como utilizando el lenguaje de consultas Flux \cite{flux} (Figura \ref{fig:data_explorer_flux})

\imagen{data_explorer}{Consulta gráfica}
\imagen{data_explorer_flux}{Consulta con Flux}

El último apartado que se explica es el de ``Dashboards'', simbolizado por cuatro cuadrados pequeños. En este apartado (Figura \ref{fig:dashboards})
podremos guardar consultas para luego visualizarlas en una especie de tablero. Se nos mostrarán todos los ``dashboards'' creados, 
y podremos crear nuevos, así como importar plantillas ya hechas (Se incluye la plantilla guardada en ``services/database/predicciones\_dashboard.json'').

\imagen{dashboards}{Apartado de ``Dashboards''}

Una vez dentro de un ``dashboard'' (Figura \ref{fig:preds}), se pueden ver las diferentes gráficas creadas en 
dicho tablero. Se puede configurar también cada cuanto quiere que se refresquen dichas gráficas y si quiere hacerse de manera 
automática (Figura \ref{fig:set_auto_refresh}).

\imagen{preds}{Tablero con las predicciones}
\imagen{set_auto_refresh}{Configuración del refresco automático.}

En esta guía solo se han mostrado aquellas herramientas importantes para el desarrollo de este proyecto. 
Información más detallada sobre todas las utilidades puede verse en la documentación de Chronograf \cite{chronograf:docs}