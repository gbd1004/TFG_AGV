\capitulo{3}{Conceptos teóricos}

% En aquellos proyectos que necesiten para su comprensión y desarrollo de unos conceptos teóricos de una determinada materia o de un determinado dominio de conocimiento, debe existir un apartado que sintetice dichos conceptos.

% Algunos conceptos teóricos de \LaTeX \footnote{Créditos a los proyectos de Álvaro López Cantero: Configurador de Presupuestos y Roberto Izquierdo Amo: PLQuiz}.

% \section{Secciones}

% Las secciones se incluyen con el comando section.

% \subsection{Subsecciones}

% Además de secciones tenemos subsecciones.

% \subsubsection{Subsubsecciones}

% Y subsecciones. 


% \section{Referencias}

% Las referencias se incluyen en el texto usando cite \cite{wiki:latex}. Para citar webs, artículos o libros \cite{koza92}.


% \section{Imágenes}

% Se pueden incluir imágenes con los comandos standard de \LaTeX, pero esta plantilla dispone de comandos propios como por ejemplo el siguiente:

% \imagen{escudoInfor}{Autómata para una expresión vacía}{.5}



% \section{Listas de items}

% Existen tres posibilidades:

% \begin{itemize}
% 	\item primer item.
% 	\item segundo item.
% \end{itemize}

% \begin{enumerate}
% 	\item primer item.
% 	\item segundo item.
% \end{enumerate}

% \begin{description}
% 	\item[Primer item] más información sobre el primer item.
% 	\item[Segundo item] más información sobre el segundo item.
% \end{description}
	
% \begin{itemize}
% \item 
% \end{itemize}

% \section{Tablas}

% Igualmente se pueden usar los comandos específicos de \LaTeX o bien usar alguno de los comandos de la plantilla.

% \tablaSmall{Herramientas y tecnologías utilizadas en cada parte del proyecto}{l c c c c}{herramientasportipodeuso}
% { \multicolumn{1}{l}{Herramientas} & App AngularJS & API REST & BD & Memoria \\}{ 
% HTML5 & X & & &\\
% CSS3 & X & & &\\
% BOOTSTRAP & X & & &\\
% JavaScript & X & & &\\
% AngularJS & X & & &\\
% Bower & X & & &\\
% PHP & & X & &\\
% Karma + Jasmine & X & & &\\
% Slim framework & & X & &\\
% Idiorm & & X & &\\
% Composer & & X & &\\
% JSON & X & X & &\\
% PhpStorm & X & X & &\\
% MySQL & & & X &\\
% PhpMyAdmin & & & X &\\
% Git + BitBucket & X & X & X & X\\
% Mik\TeX{} & & & & X\\
% \TeX{}Maker & & & & X\\
% Astah & & & & X\\
% Balsamiq Mockups & X & & &\\
% VersionOne & X & X & X & X\\
% }

Como se ha mencionado en apartados anteriores, los datos recibidos por el AGV se agrupan en series temporales. Conviene por tanto explicar
que son las series temporales, así como los modelos que se utilizarán para intentar predecirlas.


\section{Series temporales}

Una serie temporal es una colección de observaciones obtenidas mediante mediciones repetidas a lo largo del tiempo \cite{influx:timeseries}.

Normalmente, las series temporales presentan patrones que pueden utilizarse para realizar predicciones. Estos patrones son:
\begin{itemize}
    \item \textbf{Tendencia}. La tendencia existe cuando hay un incremento o decremento del valor medido a largo plazo. Esta tendencia
        no tiene por qué ser lineal.
    \item \textbf{Estacionalidad}. El patrón de estacionalidad se presenta cuando una serie temporal se ve afectada por factores estacionales,
        como puede ser el día de la semana. Siempre tiene una frecuencia fija y conocida.
    \item \textbf{Ciclos}. Un ciclo ocurre cuando los datos muestran incrementos o decrementos a una frecuencia no fija.
\end{itemize}

\imagen{pltEncDerecho.pdf}{Ejemplo de serie temporal}{1}

Existen distintos tipos de clasificaciones para las series temporales, según varios puntos de vista \cite{kitagawa2010introduction}, de
las cuales destacan:
\begin{itemize}
    \item \textbf{Continuas o discretas}. Las series temporales continuas son aquellas en las que la información se obtiene, valga la redundancia,
        de forma continua, normalmente por un dispositivo analógico, como podría ser los datos recibidos de un sismógrafo. Por otro lado,
        las series temporales discretas son aquellas en las que la información se obtiene en intervalos de tiempo concretos. Estos intervalos 
        pueden ser equidistantes, o bien ser irregulares. Normalmente, las series temporales medidas por medios digitales son discretas.
        En nuestro caso, las series temporales enviadas por el AGV son discretas espaciadas en intervalos irregulares, pues el AGV manda dicha
        información cada varios milisegundos, de manera irregular.
    \item \textbf{Univariantes o multivariantes}. Aquellas series temporales que tengan solo una observación por cada momento del tiempo son series
        univariantes. Por contra, aquellas en las que se obtengan de manera simultánea mediciones de dos o más fenómenos son multivariantes.
        Esto será importante a la hora de escoger que modelo utilizar para realizar predicciones, pues hay modelos que solo soportan series
        temporales univariantes.
    \item \textbf{Estacionarias o no estacionarias}. Una serie estacionaria \cite{hyndman2018forecasting} es aquella en la que sus propiedades no dependen del momento
        en el que se observan. Por ello, aquellas que presenten tendencias o estacionalidad no son estacionarias. Por otra parte, una serie de 
        ruido blanco es estacionaria: no importa cuándo se observe, debería tener el mismo aspecto en cualquier momento. De manera general, las
        series estacionarias no tendrán patrones predecibles en el largo plazo, por lo que conviene convertir las series estacionarias en no
        estacionarias.
\end{itemize}

\section{Predicción de series temporales}

La predicción de series temporales es una actividad muy importante en muchos sectores: predicción 
de datos financieros, predicciones del clima, etc. Debido a esto, existe una gran cantidad de 
modelos para realizar dichas predicciones. Hay que tener en cuenta sin embargo que predecir 
datos futuros es una tarea especialmente complicada, y no siempre se obtiene una gran precisión.

Los siguientes modelos se tendrán en cuenta en este trabajo:
\begin{itemize}
    \item \textbf{ARIMA.} Este modelo es una combinación de otros tres modelos: 
    \item \textbf{VARIMA.}
    \item \textbf{NHitS.}
    \item \textbf{Transformer Model.}
\end{itemize}