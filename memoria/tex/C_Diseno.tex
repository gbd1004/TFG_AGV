\apendice{Especificación de diseño}

\section{Introducción}

En esta sección se detalla la organización y flujo de datos del proyecto, el diseño de los diferentes 
servicios mencionados en apartados anteriores, y un diseño más en profundidad de toda la arquitectura.

\section{Diseño de datos}

Esta sección se divide en otras dos subsecciones: la primera detallará la organización de los datos 
en la base de datos, y la segunda detallará el flujo de datos entre los diferentes servicios.

\subsection{Base de datos}

Como se ha especificado en el anexo B, la base de datos elegida es InfluxDB. Esta es una base de datos 
de series temporales no relacional, por lo que las características de esta base de datos no siguen 
las de una base de datos relacional.

El elemento de mayor nivel organizativo en InfluxDB es la organización. Una organización es un 
entorno de trabajo en el que se definen el resto de elementos de la base de datos: usuarios, ``buckets'',
tareas, etc.

En InfluxDB, los datos se almacenan en lo que se conoce como ``buckets''. La característica principal
de estos ``buckets'' es que tienen un periodo de retención, es decir, el tiempo que perduran los datos
almacenados. Así, por ejemplo, un ``bucket'' con un periodo de retención de un minuto, los datos insertados
hace más de un minuto son automáticamente eliminados. 

Dentro de cada ``bucket'', los datos están organizados según los siguientes elementos:
\begin{enumerate}
    \item Time: Al ser InfluxDB una serie de datos de series temporales, este campo cobra 
        una gran importancia. Este campo almacena el tiempo de cada entrada según el estándar
        RFC3339 \cite{rfc3339}.
    \item Field set: El conjunto de campos, o field set, almacena pares de claves y valore: las 
        claves guardan metadatos y son cadenas de caracteres, y los valores almacenan los datos 
        propiamente dichos. 
    \item Tag set: el conjunto de etiquetas, o tag set, almacena, al igual que el field set, pares 
        de clave-valor. El tag set, sin embargo, solo almacena metadatos, y tanto la clave como el
        valor almacenan cadenas de caracteres. Este elemento es opcional, aunque es muy recomendado 
        utilizarlo, pues la etiquetas están indexadas, por lo que las consultas en este tipo de datos 
        son mucho más rápidas.
    \item Measurement: la medida, o measurement, actúa como contenedor de los campos anteriores, 
        y sería el equivalente a una tabla en una base de datos relacional.
\end{enumerate}

Con esta información en mente, la organización de la base de datos queda definida de la siguiente 
manera. Se utilizan dos ``buckets'', uno para los datos recibidos por el AGV o por el simulador con 
un periodo de retención infinito (los datos no se eliminan nunca), y otro para almacenar las predicciones 
realizadas con un periodo de retención bajo, especificado en un archivo de configuración. 

En cada uno de estos ``buckets'' se almacenan las series temporales siguiendo el esquema de las 
tablas \ref{tabla:bucket_agv} y \ref{tabla:bucket_pred}

\tablaSmallSinColores{Bucket AGV}{l l l}{bucket_agv}{
    Elemento & Descripción \\
}{
    Measurement & \multicolumn{2}{l}{AGVDATA} \\
    \midrule
    & Clave & Valor \\
    Tag set & agvid & string \\
    & type & string \\
    \midrule
    & Clave & Valor \\
    \multirow{9}{*}{Field set} & encoder\_derecho & int \\
    & encoder\_izquierdo & int \\
    & in.current\_l & int \\
    & in.current\_h & int \\
    & in.i\_medida\_bat & int \\
    & in.guide\_error & float \\
    & out.set\_speed\_right & int \\
    & out.set\_speed\_left & int \\
    & out.display & int \\
}

El bucket AGV contiene un tag para la ID del AGV que mande dichos datos. Esto ahora mismo no se utiliza, pues 
solo se puede simular y predecir un único AGV, pero se incluye por si en el futuro se decide ampliar para poder 
trabajar con varios vehículos. 

La otra tag, ``type'', se incluye también para posibles futuros desarrollos. Todos los valores enviados por los 
vehículos o por el simulador deben tener el valor ``value''. En el futuro, en caso de que quieran almacenarse 
otros datos externos a los vehículos, o datos procesados de los mismos, se han de introducir con otro valor de 
esta tag diferente.

\tablaSmallSinColores{Bucket Predictions}{l l l}{bucket_pred}{
    Elemento & Descripción \\
}{
    Measurement & \multicolumn{2}{l}{pred} \\
    \midrule
    & Clave & Valor \\
    \multirow{2}{*}{Field set} & encoder\_derecho & int \\
    & encoder\_izquierdo & int \\
}

El bucket de las predicciones es mucho más simple. Simplemente tiene los valores generados por el servicio de 
predicciones.

\subsection{Flujo de datos}

La comunicación entre servicios ocurre de tres formas diferentes:
\begin{itemize}
    \item 5G Wifi: Los AGV envían datos al ``AGV Coordinator'' como cadenas de bytes a este servicio.
        Como este servicio no está desarrollado como parte de este proyecto, no se detallará más.
    \item UDP/JSON: Tanto el ``AGV Coordinator'' como el ``Simulator'' envían los datos codificados en 
        forma de JSON a través de una conexión UDP. Estos datos son recibidos por el servicio ``Reciever'',
        que los introducirá en la base de datos.
    \item Database API: El ``Reciever'' es el encargado de insertar los datos de los AGV o del simulador, 
        utilizando para ello la API de la base de datos a través de consultas.
\end{itemize}

\section{Diseño procedimental}

% TODO

\section{Diseño arquitectónico}

% TODO

Debido a la simplicidad del código utilizado en el proyecto, se ha decidido seguir un enfoque de programación 
funcional, más que programación orientada a objetos. Por esto, no podrán definirse diagramas típicos de UML como 
puede ser un diagrama de clases.

\subsection{Diseño de paquetes}

El código del proyecto ha sido organizado según el servicio al que pertenecen. De esta forma se simplifica la organización,
consiguiendo un proyecto con un código más fácil de leer y de mantener.

\imagenSized{diagrama_paquetes}{Diagrama de paquetes}{0.45}

Como se puede ver en la Figura \ref{fig:diagrama_paquetes}, existen cuatro paquetes, uno por cada servicio:
\begin{itemize}
    \item Database: contiene los scripts que se ejecutan al iniciar la base de datos.
    \item Simulator: contiene el código y archivos de configuración relacionados con el simulador.
    \item Reciever: contiene el código y archivos de configuración del servicio ``Reciever''.
    \item Forecasting: contiene el código y archivos de configuración del servicio de predicción.
\end{itemize}

\subsection{Diseño de despliegue}

Un diagrama de despliegue muestra la arquitectura de ejecución de sistemas que representan la asignación, o despliegue,
de artefactos de software a objetivos de despliegue \cite{uml:deploy}. La Figura \ref{fig:deploy_diag} muestra el 
diagrama de despliegue del sistema.

Cada nodo normalmente representa un dispositivo hardware o un entorno de ejecución de software, y las asociaciones 
representan las comunicaciones entre nodos.

\imagenSized{deploy_diag}{Diagrama de despliegue}{0.8}