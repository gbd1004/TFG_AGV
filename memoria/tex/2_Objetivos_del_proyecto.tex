\capitulo{2}{Objetivos del proyecto}

El objetivo principal de este trabajo es desarrollar un sistema capaz de almacenar, visualizar y predecir 
datos producidos por un AGV. Este sistema está planteado para permitir la realización de mantenimiento 
predictivo o alertas de fallos en el funcionamiento de manera automática en entornos industriales, 
por lo que su objetivo final será el de reducir costes y mejorar la eficiencia en estos entornos.

Se pretendo conseguir también un sistema eficiente, por lo que será necesario una extensa investigación 
y comparación de sistemas gestores de bases de datos a utilizar, así como de diferentes modelos predictivos.

Este proyecto ha sido diseñado como un conjunto de microservicios \cite{7030212}. Las aplicaciones diseñadas de esta manera se dividen 
en elementos más pequeños e independientes entre sí, y que deben comunicarse y coordinarse. De esta 
forma, se consigue que la aplicación sea modular, más fácil de escalar, de mantener y de desarrollar.

Al dividir la aplicación en lo que a efectos prácticos son aplicaciones más simples, se consigue un código 
mucho más sencillo, por lo que el diseño del propio código se simplifica. En nuestro caso, se ha utilizado 
un enfoque de programación procedimental, en vez de programación orientada a objetos, pues la simplicidad del propio 
código hace que esta última metodología no sea la más apropiada.

Los objetivos pueden quedar resumidos en la siguiente lista:
\begin{itemize}
    \item Desarrollar sistema que permita la monitorización de un sistema de AGVs enviando información en tiempo real.
    \item Diseñar una metodología que sirva como base para realizar una comparativa de sistemas gestores de bases de datos.
    \item Crear un sistema que almacene de manera eficiente la información, utilizando para ello el sistema gestor de bases 
        de datos que mejor se adapte a los requisitos según la metodología diseñada.
    \item Realizar predicciones sobre los datos almacenados a diez segundos en el futuro.
    \item Diseñar una metodología para realizar una comparativa de modelos de predicción de datos, de manera que se escoja el 
        más eficiente.
    \item Desarrollar el sistema como un conjunto de microservicios que interaccionan entre sí para conseguir una
        mayor modularidad.
    \item Diseñar un código mantenible y que se adapte a las convenciones estilísticas establecidas a partir del uso de
        herramientas de integración continua como CodeClimate o Pylint.
    \item Mantener un control de las versiones del proyecto a partir de un uso correcto de Git.
    \item Desarrollar el sistema de manera organizada aplicando la metodología ágil.
\end{itemize}