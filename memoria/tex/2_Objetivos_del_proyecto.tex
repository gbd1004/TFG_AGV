\capitulo{2}{Objetivos del proyecto}

El objetivo principal de este trabajo es el de desarrollar un sistema dedicado a un entorno profesional, capaz 
de mejorar la eficiencia y reducir costes en procesos industriales.

Este proyecto ha sido diseñado como un microservicio. Las aplicaciones diseñadas de esta manera se dividen 
en elementos más pequeños e independientes entre sí, y que deben comunicarse y coordinarse. De esta 
forma, se consigue que la aplicación sea modular, más facil de escalar, de mantener y de desarrollar.

Al dividir la aplicación en lo que a efectos prácticos son aplicaciones más simples, se consigue un código 
mucho más sencillo, por lo que el diseño del propio código se simplifica. En nuestro caso, se ha utilizado 
un enfoque de programación funcional, en vez de programación orientada a objetos, pues la simplicidad del propio 
código hace que esta última metodología no sea la más apropiada.

Los objetivos pueden quedar resumidos en la siguiente lista:
\begin{itemize}
    \item Desarrollar la aplicación como un microservicio.
    \item Utilizar Docker para agrupar cada servicio en su contenedor independiente.
    \item Utilizar un enfoque de programación funcional.
    \item Realizar un estudio comparativo de sistemas gestores de bases de datos.
    \item Realizar otro estudio comparando modelos de predicción de series temporales.
    \item Hacer uso de Git como herramienta de control de versiones.
    \item Utilizar herramientas de integración continua como CodeClimate.
    \item Aplicar la metodología ágil durante el desarrollo del software.
\end{itemize}