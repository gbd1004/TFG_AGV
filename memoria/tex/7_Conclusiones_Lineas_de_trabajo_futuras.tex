\capitulo{7}{Conclusiones y líneas de trabajo futuras}

A continuación, se exponen las conclusiones obtenidas tras la finalización del proyecto, así como mejoras a realizar 
en el futuro.

\section{Conclusiones}

El objetivo general del proyecto ha sido cumplido de manera satisfactoria: se ha conseguido el desarrollo de un sistema
capaz de almacenar los datos recibidos por un AGV o por el simulador y de llevar a cabo predicciones precisas que permitan 
realizar mantenimiento predictivo.

Se ha desarrollado un sistema modular, basado en el desarrollo de microservicios independientes entre sí. Gracias a
esto, se ha conseguido también que el sistema se adapte a las limitaciones de hardware del cliente.

Gracias a la utilización de Python y Docker, se ha conseguido crear un sistema sobre el que es muy fácil de realizar 
modificaciones a futuro: Python agiliza mucho el desarrollo al ser un lenguaje interpretado, que no necesita de compilaciones 
que resten tiempo al proceso de despliegue, y Docker permite modificar solo aquellos servicios en los que haya cambios.

Durante el desarrollo del proyecto, se han utilizado metodologías y conocimientos obtenidos durante el grado. Se 
ha seguido una metodología ágil para el desarrollo, se han aprovechado los conocimientos obtenidos sobre bases de 
datos para la elección del sistema gestor de bases de datos, se han utilizado los conocimientos obtenidos sobre 
inteligencia artificial y aprendizaje automático, etc.

El desarrollo del proyecto ha servido también para ampliar mis conocimientos en dichos campos. Me ha servido 
también para adquirir nuevos conocimientos, como herramientas de Integración continua, nuevos métodos de aprendizaje 
automático, uso de contenedores de Docker para la creación de microservicios, etc.

Por el trabajo de investigación realizado durante el desarrollo del proyecto, mi capacidad para buscar información 
de calidad ha mejorado de manera sustancial, así como mi capacidad para leer y redactar documentos científicos.

Quiero destacar también la labor realizada por mis tutores, guiándome de manera efectiva durante todo el desarrollo del 
trabajo. También, por sugerencia suya, una parte de este trabajo ha sido publicado en la conferencia SOCO 2023 \cite{8364SOCO}.

\section{Líneas de trabajo futuras}


Con el fin de mejorar la modularidad del proyecto, se sugieren los siguientes cambios:
\begin{itemize}
    \item Crear un nuevo servicio que haga de intermediario para otros servicios que quieran interaccionar con la propia base de datos 
        (Figura \ref{fig:arquitectura_futura}). Para ello, se propone el desarrollado de una API Rest, mediante la cual 
        se hagan las peticiones a este servicio para hacer consultas e inserciones. De esta forma, se consigue 
        desacoplar los diferentes servicios de la base de datos, lo que resultaría en un diseño mucho más modular, y 
        que simplificaría mucho el desarrollo de nuevos servicios en el futuro.
    \item Modificar el servicio ``Forecaster'', para que pueda detectar comportamientos anómalos de los AGV de manera 
        automática utilizando la detección de anomalías de Darts \cite{darts:anomaly}, herramienta utilizada para realizar las predicciones.
    \item Modificar el servicio ``Simulator'' de manera que permita simular más de un AGV. De manera similar, se sugiere 
        modificar también el servicio ``Forecaster'' para que pueda realizar predicciones de varios vehículos.
    \item Ejecutar la optimización de los modelos de predicción comparados durante más tiempo. Por limitaciones 
        de plazos, no fue posible optimizar dichos modelos durante mucho tiempo, por lo que todavía se pueden obtener 
        mejores modelos potenciales.
\end{itemize}

\imagen{arquitectura_futura}{Diseño futuro de la arquitectura}{1}