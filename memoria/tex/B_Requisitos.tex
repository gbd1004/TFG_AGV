\apendice{Especificación de Requisitos}

\section{Introducción}

En esta sección de los anexos se especifican los requisitos, tanto funcionales como no funcionales, que el sistema 
debe de cumplir.

Se incluye en esta sección también la comparativa entre sistemas gestores de bases de datos y modelos de predicción 
acordes a dichos requisitos.

\section{Objetivos generales}

El proyecto ha de cumplir los siguientes objetivos generales:
\begin{enumerate}
	\item Desarrollar un sistema capaz de almacenar, monitorizar y predecir los datos enviados por los AGV.
	\item Desarrollar los servicios del sistema de la forma más modular posible, de forma que simplifique la modificación de las
		funcionalidades del sistema.
	\item Realizar un estudio de las herramientas a utilizar con el fin de obtener el sistema más eficiente posible.
	\item El funcionamiento de los servicios que componen el sistema han de ser configurables a través de archivos de
		configuración.
\end{enumerate}

\section{Catálogo de requisitos}

\subsection{Requisitos funcionales}

Se muestran a continuación los requisitos obtenidos a partir del análisis de los objetivos generales mencionados anteriormente.
\begin{itemize}
	\item \textbf{RF-1 Gestión de datos:} el sistema tiene que ser capaz de almacenar los datos enviados por el AGV.
	\begin{itemize}
		\item \textbf{RF-1.1 Visualización de datos:} el usuario debe poder ser capaz de visualizar los datos del AGV
		\item \textbf{RF-1.2 Visualización de predicciones:} el usuario tiene que ser capaz de ver las predicciones realizadas.
	\end{itemize}
	\item \textbf{RF-2 Predicción de datos:} el sistema tiene que poder generar predicciones a partir de los datos almacenados.
	\begin{itemize}
		\item \textbf{RF-2.1 Configuración de entrenamiento: } el usuario debe ser capaz de especificar en un archivo de configuración
			si se quiere entrenar un nuevo modelo o cargarlo desde un archivo.
		\item \textbf{RF-2.2 Cargar modelo desde archivo: } el usuario debe ser capaz de especificar en un archivo de configuración
			el nombre del archivo del modelo a cargar.
	\end{itemize}
	\item \textbf{RF-3 Simulación de datos:} el sistema ha de poder simular datos en caso de no disponer de un AGV.
	\begin{itemize}
		\item \textbf{RF-3.1 Simulación aleatoria:} el simulador tiene que ser capaz de generar datos aleatorios.
		\item \textbf{RF-3.2 Simulación desde CSV:} el simulador debe poder leer los datos desde un archivo CSV con
			información real de un AGV.
		\item \textbf{RF-3.3 Desactivado desde configuración:} el usuario debe ser capaz de activar o desactivar el simulador 
			desde un archivo de configuración.
	\end{itemize}
\end{itemize}

\subsection{Requisitos no funcionales}
\begin{itemize}
	\item \textbf{RNF-1 Modularidad:} el sistema ha de desarrollarse de forma que los servicios que lo integran sean lo más 
		modulares posible.
	\item \textbf{RNF-2 Escalabilidad:} el sistema ha de estar desarrollado de manera que permita ser escalable en el futuro.
	\item \textbf{RNF-3 Licencias:} todas las herramientas utilizadas deben ser de código abierto.
	\item \textbf{RNF-4 Compatibilidad de hardware:} el sistema ha de soportar aquellos sistemas con una GPU Nvidia con soporte para CUDA
		11.8. Para aquellos sistemas que no cumplan este requisito, se ha de poder realizar el entrenamiento de los modelos
		usando la CPU.
	\item \textbf{RNF-5 Compatibilidad de software:} todas las herramientas software utilizadas han de tener un buen soporte
		para Python y GNU/Linux.
	\item \textbf{RNF-6 Rendimiento:} el sistema tiene que ser capaz de almacenar los datos del AGV en tiempo real, así como
		tener un buen rendimiento a la hora de realizar las predicciones.
	\item \textbf{RNF-7 Momento de inserción:} las entradas de la base de datos han de poder insertarse con el timestamp del momento
		en el que dichos datos fueron generados en el AGV. 
    \item \textbf{RNF-8 Precisión temporal} El timestamp de cada entrada debe tener una precisión en el rango de
        los milisegundos.
	\item \textbf{RNF-9 Precisión:} los datos tienen que poder insertarse en cualquier momento a la base de datos.
	\item \textbf{RNF-10 Predicción:} el modelo de predicción ha de generar resultados aceptables 10 segundos en el futuro.
\end{itemize}

\section{Especificación de requisitos}

\subsection{Diagrama de casos de uso}

Se muestra a continuación el diagrama de casos de uso (Figura \ref{fig:casos_uso})

\imagen{casos_uso}{Diagrama de casos de uso}

\subsection{Casos de uso detallados}
% TODO revisar formato para que las tablas se "rompan" con un page break
A continuación se detallan los diferentes casos de uso:

\begin{table}[H]
	\centering
	\begin{tabularx}{\linewidth}{ p{0.21\columnwidth} p{0.71\columnwidth} }
		\toprule
		\textbf{CU-1}    & \textbf{Gestión de datos}\\
		\toprule
		\textbf{Versión}              & 1.0    \\
		\textbf{Autor}                & Gonzalo Burgos de la Hera \\
		\textbf{Requisitos asociados} & RF-1, RF-1.1, RF-1.2 \\
		\textbf{Descripción}          & El sistema tiene que ser capaz de almacenar los datos enviados por el AGV. \\
		\textbf{Precondición}         & La base de datos debe estar activa. \\
		\textbf{Acciones}             &
		\begin{enumerate}
			\def\labelenumi{\arabic{enumi}.}
			\tightlist
			\item El AGV o el simulador envían datos.
			\item El servicio ``Receiver'' envía esos datos a la base de datos.
		\end{enumerate}\\
		\textbf{Postcondición}        & Los datos se almacenan. \\
		\textbf{Excepciones}          & 
        \begin{itemize}
			\tightlist
			\item No se puede conectar con la base de datos.
			\item El servicio ``Receiver'' no está activo.
        \end{itemize} \\
		\textbf{Importancia}          & Alta \\
		\bottomrule
	\end{tabularx}
	\caption{CU-1 Gestión de datos.}
\end{table}

\begin{table}[H]
	\centering
	\begin{tabularx}{\linewidth}{ p{0.21\columnwidth} p{0.71\columnwidth} }
		\toprule
		\textbf{CU-1}    & \textbf{Visualización de datos}\\
		\toprule
		\textbf{Versión}              & 1.0    \\
		\textbf{Autor}                & Gonzalo Burgos de la Hera \\
		\textbf{Requisitos asociados} & RF-1.1 \\
		\textbf{Descripción}          & El usuario tiene que ser capaz de visualizar los datos almacenados. \\
		\textbf{Precondición}         & La base de datos debe estar activa. \\
		\textbf{Acciones}             &
		\begin{enumerate}
			\def\labelenumi{\arabic{enumi}.}
			\tightlist
			\item El usuario hace la consulta determinada a la base de datos.
		\end{enumerate}\\
		\textbf{Postcondición}        & Los datos se visualizan. \\
		\textbf{Excepciones}          & 
        \begin{itemize}
			\tightlist
			\item No se puede conectar con la base de datos.
        \end{itemize} \\
		\textbf{Importancia}          & Alta \\
		\bottomrule
	\end{tabularx}
	\caption{CU-2 Visualización de datos.}
\end{table}

\begin{table}[H]
	\centering
	\begin{tabularx}{\linewidth}{ p{0.21\columnwidth} p{0.71\columnwidth} }
		\toprule
		\textbf{CU-1}    & \textbf{Visualización de predicciones}\\
		\toprule
		\textbf{Versión}              & 1.0    \\
		\textbf{Autor}                & Gonzalo Burgos de la Hera \\
		\textbf{Requisitos asociados} & RF-1.2 \\
		\textbf{Descripción}          & El usuario tiene que ser capaz de visualizar las predicciones realizadas. \\
		\textbf{Precondiciones}       & 
        \begin{enumerate}
			\def\labelenumi{\arabic{enumi}.}
			\tightlist
			\item La base de datos debe estar activa.
			\item El servicio de predicción está activo.
		\end{enumerate}\\
		\textbf{Acciones}             &
		\begin{enumerate}
			\def\labelenumi{\arabic{enumi}.}
			\tightlist
			\item El usuario hace la consulta determinada a la base de datos.
		\end{enumerate}\\
		\textbf{Postcondición}        & Las predicciones se visualizan. \\
		\textbf{Excepciones}          & 
        \begin{itemize}
			\tightlist
			\item No se puede conectar con la base de datos.
			\item El sistema de predicción no está activo.
        \end{itemize} \\
		\textbf{Importancia}          & Alta \\
		\bottomrule
	\end{tabularx}
	\caption{CU-3 Visualización de datos.}
\end{table}

\begin{table}[H]
	\centering
	\begin{tabularx}{\linewidth}{ p{0.21\columnwidth} p{0.71\columnwidth} }
		\toprule
		\textbf{CU-1}    & \textbf{Predicción de datos}\\
		\toprule
		\textbf{Versión}              & 1.0    \\
		\textbf{Autor}                & Gonzalo Burgos de la Hera \\
		\textbf{Requisitos asociados} & RF-2, RF-2.1, RF-2.2 \\
		\textbf{Descripción}          & El sistema tiene que poder generar predicciones a partir de los datos almacenados. \\
		\textbf{Precondiciones}       & 
        \begin{enumerate}
			\def\labelenumi{\arabic{enumi}.}
			\tightlist
			\item La base de datos debe estar activa.
			\item El servicio de predicción está activo.
		\end{enumerate}\\
		\textbf{Acciones}             &
		\begin{enumerate}
			\def\labelenumi{\arabic{enumi}.}
			\tightlist
			\item El servicio de predicciones hace una consulta con los datos necesarios.
			\item El servicio de predicciones realiza la predicción.
			\item El servicio inserta la predicción en la base de datos.
		\end{enumerate}\\
		\textbf{Postcondición}        & Las predicciones se almacenan en la base de datos. \\
		\textbf{Excepciones}          & 
        \begin{itemize}
			\tightlist
			\item No se puede conectar con la base de datos.
			\item El sistema de predicción no está activo.
        \end{itemize} \\
		\textbf{Importancia}          & Alta \\
		\bottomrule
	\end{tabularx}
	\caption{CU-4 Predicción de datos.}
\end{table}

\begin{table}[H]
	\centering
	\begin{tabularx}{\linewidth}{ p{0.21\columnwidth} p{0.71\columnwidth} }
		\toprule
		\textbf{CU-1}    & \textbf{Configuración de entrenamiento}\\
		\toprule
		\textbf{Versión}              & 1.0    \\
		\textbf{Autor}                & Gonzalo Burgos de la Hera \\
		\textbf{Requisitos asociados} & RF-2.1 \\
		\textbf{Descripción}          & El usuario debe ser capaz de especificar en un archivo de configuración si se quiere entrenar un nuevo modelo o cargarlo desde un archivo. \\
		\textbf{Precondiciones}       & 
        \begin{enumerate}
			\def\labelenumi{\arabic{enumi}.}
			\tightlist
			\item El servicio de predicción está inactivo.
		\end{enumerate}\\
		\textbf{Acciones}             &
		\begin{enumerate}
			\def\labelenumi{\arabic{enumi}.}
			\tightlist
			\item El usuario modifica la configuración.
		\end{enumerate}\\
		\textbf{Postcondición}        & El servicio de predicciones se inicia con la configuración especificada. \\
		\textbf{Excepciones}          & 
        \begin{itemize}
			\tightlist
			\item Se introduce una configuración inválida.
        \end{itemize} \\
		\textbf{Importancia}          & Alta \\
		\bottomrule
	\end{tabularx}
	\caption{CU-5 Configuración de entrenamiento.}
\end{table}

\begin{table}[H]
	\centering
	\begin{tabularx}{\linewidth}{ p{0.21\columnwidth} p{0.71\columnwidth} }
		\toprule
		\textbf{CU-1}    & \textbf{Cargar modelo desde archivo}\\
		\toprule
		\textbf{Versión}              & 1.0    \\
		\textbf{Autor}                & Gonzalo Burgos de la Hera \\
		\textbf{Requisitos asociados} & RF-2.2 \\
		\textbf{Descripción}          & El usuario debe ser capaz de especificar en un archivo de configuración el nombre del archivo del modelo a cargar. \\
		\textbf{Precondiciones}       & 
        \begin{enumerate}
			\def\labelenumi{\arabic{enumi}.}
			\tightlist
			\item El servicio de predicción está inactivo.
		\end{enumerate}\\
		\textbf{Acciones}             &
		\begin{enumerate}
			\def\labelenumi{\arabic{enumi}.}
			\tightlist
			\item El usuario modifica la configuración.
		\end{enumerate}\\
		\textbf{Postcondición}        & El servicio de predicciones se inicia con la configuración especificada. \\
		\textbf{Excepciones}          & 
        \begin{itemize}
			\tightlist
			\item Se introduce una configuración inválida.
        \end{itemize} \\
		\textbf{Importancia}          & Alta \\
		\bottomrule
	\end{tabularx}
	\caption{CU-6 Cargar modelo desde archivo.}
\end{table}

\begin{table}[H]
	\centering
	\begin{tabularx}{\linewidth}{ p{0.21\columnwidth} p{0.71\columnwidth} }
		\toprule
		\textbf{CU-1}    & \textbf{Simulación de datos}\\
		\toprule
		\textbf{Versión}              & 1.0    \\
		\textbf{Autor}                & Gonzalo Burgos de la Hera \\
		\textbf{Requisitos asociados} & RF-3, RF-3.1, RF-3.2, RF-3.3 \\
		\textbf{Descripción}          & El sistema ha de poder simular datos en caso de no disponer de un AGV.\\
		\textbf{Precondiciones}       & 
        \begin{enumerate}
			\def\labelenumi{\arabic{enumi}.}
			\tightlist
			\item El servicio de simulación está activo en la configuración.
		\end{enumerate}\\
		\textbf{Acciones}             &
		\begin{enumerate}
			\def\labelenumi{\arabic{enumi}.}
			\tightlist
			\item El simulador genera datos.
			\item El servicio ``Receiver'' recibe los datos.
		\end{enumerate}\\
		\textbf{Postcondición}        & Los datos se almacenan en la base de datos. \\
		\textbf{Excepciones}          & 
        \begin{itemize}
			\tightlist
			\item El simulador no se inicia.
        \end{itemize} \\
		\textbf{Importancia}          & Alta \\
		\bottomrule
	\end{tabularx}
	\caption{CU-7 Simulación de datos.}
\end{table}

\begin{table}[H]
	\centering
	\begin{tabularx}{\linewidth}{ p{0.21\columnwidth} p{0.71\columnwidth} }
		\toprule
		\textbf{CU-1}    & \textbf{Simulación aleatoria}\\
		\toprule
		\textbf{Versión}              & 1.0    \\
		\textbf{Autor}                & Gonzalo Burgos de la Hera \\
		\textbf{Requisitos asociados} & RF-3.1 \\
		\textbf{Descripción}          & El simulador tiene que ser capaz de generar datos aleatorios.\\
		\textbf{Precondiciones}       & 
        \begin{enumerate}
			\def\labelenumi{\arabic{enumi}.}
			\tightlist
			\item El servicio de simulación está activo en la configuración.
			\item El servicio de simulación está configurado para generar datos aleatorios.
		\end{enumerate}\\
		\textbf{Acciones}             &
		\begin{enumerate}
			\def\labelenumi{\arabic{enumi}.}
			\tightlist
			\item El simulador genera datos.
			\item El servicio ``Receiver'' recibe los datos.
		\end{enumerate}\\
		\textbf{Postcondición}        & Los datos se almacenan en la base de datos. \\
		\textbf{Excepciones}          & 
        \begin{itemize}
			\tightlist
			\item El simulador no se inicia.
        \end{itemize} \\
		\textbf{Importancia}          & Alta \\
		\bottomrule
	\end{tabularx}
	\caption{CU-8 Simulación aleatoria.}
\end{table}

\begin{table}[H]
	\centering
	\begin{tabularx}{\linewidth}{ p{0.21\columnwidth} p{0.71\columnwidth} }
		\toprule
		\textbf{CU-1}    & \textbf{Simulación desde CSV}\\
		\toprule
		\textbf{Versión}              & 1.0    \\
		\textbf{Autor}                & Gonzalo Burgos de la Hera \\
		\textbf{Requisitos asociados} & RF-3.2 \\
		\textbf{Descripción}          & El simulador debe poder leer los datos desde un archivo CSV con información real de un AGV. \\
		\textbf{Precondiciones}       & 
        \begin{enumerate}
			\def\labelenumi{\arabic{enumi}.}
			\tightlist
			\item El servicio de simulación está activo en la configuración.
			\item El servicio de simulación está configurado para generar datos desde un CSV.
		\end{enumerate}\\
		\textbf{Acciones}             &
		\begin{enumerate}
			\def\labelenumi{\arabic{enumi}.}
			\tightlist
			\item El simulador genera datos.
			\item El servicio ``Receiver'' recibe los datos.
		\end{enumerate}\\
		\textbf{Postcondición}        & Los datos se almacenan en la base de datos. \\
		\textbf{Excepciones}          & 
        \begin{itemize}
			\tightlist
			\item El simulador no se inicia.
			\item El archivo CSV especificado no existe.
        \end{itemize} \\
		\textbf{Importancia}          & Alta \\
		\bottomrule
	\end{tabularx}
	\caption{CU-9 Simulación desde CSV.}
\end{table}

\begin{table}[H]
	\centering
	\begin{tabularx}{\linewidth}{ p{0.21\columnwidth} p{0.71\columnwidth} }
		\toprule
		\textbf{CU-1}    & \textbf{Desactivado desde configuración}\\
		\toprule
		\textbf{Versión}              & 1.0    \\
		\textbf{Autor}                & Gonzalo Burgos de la Hera \\
		\textbf{Requisitos asociados} & RF-3.3 \\
		\textbf{Descripción}          & El usuario debe ser capaz de activar o desactivar el simulador desde un archivo de configuración. \\
		\textbf{Precondiciones}       & 
        \begin{enumerate}
			\def\labelenumi{\arabic{enumi}.}
			\tightlist
            \item El servicio de simulación no está iniciado.
			\item El servicio de simulación está desactivado en la configuración.
		\end{enumerate}\\
		\textbf{Acciones}             &
		\begin{enumerate}
			\def\labelenumi{\arabic{enumi}.}
			\tightlist
			\item Se inicia el simulador.
			\item Inmediatamente después el simulador termina su ejecución.
		\end{enumerate}\\
		\textbf{Postcondición}        & El simulador se detiene. \\
		\textbf{Excepciones}          & 
        \begin{itemize}
			\tightlist
			\item Se introduce una configuración inválida.
        \end{itemize} \\
		\textbf{Importancia}          & Alta \\
		\bottomrule
	\end{tabularx}
	\caption{CU-10 Desactivado desde configuración.}
\end{table}

