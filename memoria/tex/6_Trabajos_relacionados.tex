\capitulo{6}{Trabajos relacionados}

En un entorno industrial, aumentar la eficiencia de los sistemas utilizados es un proceso esencial, pues permite 
mejorar la productividad de los procesos y reducir costes. Por ello, hay una gran cantidad de artículos científicos 
que tratan temas de mantenimiento predictivo utilizando técnicas como el aprendizaje automático.

A continuación, se muestran varios artículos que tratan este problema, ya sea en entornos en los que se utilicen
AGV, o entornos puramente industriales.

\section{Comparativas de sistemas gestores de bases de datos}

\paragraph{A Comparison Of Relational, NoSQL and NewSQL Database Management Systems For The Persistence Of Time Series Data}

Este trabajo \cite{9988333} evalúa diferentes sistemas de gestión de bases de datos en el contexto del almacenamiento de 
datos relacionados con el tiempo utilizando diferentes modelos de datos, como modelos relacionales clásicos, modelos no 
relacionales que utilizan sistemas de bases de datos NoSQL y el grupo de bases de datos NewSQL de reciente aparición. 
La evaluación muestra que una base de datos de series temporales altamente optimizada como InftuxDB es capaz de superar 
a los otros sistemas probados en cuanto a rendimiento de escritura y utilización de RAM y disco en una configuración de 
servidor único.

\paragraph{Evaluation of modern tools and techniques for storing time-series data}

Según los autores \cite{STRUCKOV201919}, este trabajo se centra en la importancia creciente del análisis y las aplicaciones 
de los datos de series temporales en diversas áreas y dominios. Se menciona que muchos campos científicos e industriales 
dependen del almacenamiento y procesamiento de grandes cantidades de series temporales, como la economía y las finanzas, 
la medicina, Internet de las Cosas, la protección ambiental, el monitoreo de hardware, entre otros. El objetivo de este 
trabajo es presentar un enfoque teórico y experimental para seleccionar una herramienta apropiada para el análisis de 
series temporales.

\section{Modelos de predicción en entornos industriales}

\paragraph{Predictive maintenance enabled by machine learning: Use cases and challenges in the automotive industry}
En este artículo \cite{THEISSLER2021107864}, se realiza un estudio y clasificación de estudios que proponen la aplicación de aprendizaje 
automático para la realización de mantenimiento predictivo de en el campo de la industria automovilística.
Se proponen retos a resolver, así como posibles direcciones de investigación. Las conclusiones obtenidas finalmente 
son que disponer de datos públicos impulsaría las actividades de investigación, la mayoría de los artículos se basan 
en métodos supervisados que requieren datos etiquetados, la combinación de múltiples fuentes de datos puede 
mejorar la precisión, y el uso de métodos de aprendizaje profundo seguirá aumentando, pero requiere métodos 
eficientes e interpretables y la disponibilidad de grandes cantidades de datos.

\paragraph{Availability assessment for a multi-AGV system based on simulation modeling approach}
Este artículo \cite{9590979} propone el desarrollo de una simulación de Monte Carlo de un sistema multi-AGV 
utilizado para evaluar la disponibilidad del sistema. El análisis de sensibilidad para investigar las relaciones 
directas entre el mantenimiento y los parámetros operativos y el nivel de ratio de disponibilidad se 
realiza basándose en el modelo desarrollado. El análisis detallado de los logros en este ámbito permite 
identificar las lagunas en la investigación y las posibles líneas de investigación futuras para los 
procesos de diseño de almacenes autónomos.

\paragraph{Combining empirical mode decomposition and deep recurrent neural networks for predictive maintenance of lithium-ion battery}
Los autores proponen \cite{CHEN2021101405} un método híbrido de ciencia de datos basado en la descomposición empírica de modos, el análisis relacional gris 
y las redes neuronales recurrentes profundas para la predicción de la vida útil restante de las baterías de iones de litio.
Los resultados experimentales obtenidos con los datos de las baterías de iones de litio del Repositorio de 
Datos de Prognosis Ames de la NASA muestran que el modelo híbrido de ciencia de datos propuesto puede 
predecir con precisión el estado de salud y la vida útil restante de las baterías de iones de litio.

\paragraph{Novel methodology for optimising the design, operation and maintenance of a multi-AGV system}
Según los autores \cite{YAN2018130}, los problemas de fiabilidad y las estrategias de mantenimiento de los AGV no se han estudiado suficientemente,
por lo que han realizado una investigación considerando un sistema multi-AGV, compuesto por tres AGVs, con 
el fin de desarrollar una metodología científica para optimizar el diseño del layout, la operación y el 
mantenimiento de un sistema multi-AGV. Los resultados de simulación obtenidos muestran claramente que la 
ubicación de los puntos de mantenimiento y las estrategias de mantenimiento tienen una influencia 
significativa en el rendimiento de un sistema multi-AGV, donde el mantenimiento correctivo es una medida 
eficaz para mantener la fiabilidad y estabilidad del sistema a largo plazo.

\section{Otros proyectos}

Si bien existen soluciones como Amazon Forecast encargadas de hacer predicciones sobre datos, hasta donde se ha podido observar, 
no existen soluciones de software dedicadas a la predicción de comportamientos específicamente de AGV, por lo que este proyecto 
ofrece un sistema original e innovador en su campo.